%%%%%%%%%%%%%%%%%%%%%%%%%%%%%%%%%%%%%%%%%
% a0poster Landscape Poster
% LaTeX Template
% Version 1.0 (22/06/13)
%
% The a0poster class was created by:
% Gerlinde Kettl and Matthias Weiser (tex@kettl.de)
% 
% This template has been downloaded from:
% http://www.LaTeXTemplates.com
%
% License:
% CC BY-NC-SA 3.0 (http://creativecommons.org/licenses/by-nc-sa/3.0/)
%
%%%%%%%%%%%%%%%%%%%%%%%%%%%%%%%%%%%%%%%%%

%----------------------------------------------------------------------------------------
%	PACKAGES AND OTHER DOCUMENT CONFIGURATIONS
%----------------------------------------------------------------------------------------

\documentclass[a0,landscape]{a0poster}

\usepackage{multicol} % This is so we can have multiple columns of text side-by-side
\columnsep=100pt % This is the amount of white space between the columns in the poster
\columnseprule=3pt % This is the thickness of the black line between the columns in the poster
%\raggedcolumns
\usepackage{fontawesome}

\usepackage[svgnames]{xcolor} % Specify colors by their 'svgnames', for a full list of all colors available see here: http://www.latextemplates.com/svgnames-colors

\usepackage{times} % Use the times font
%\usepackage{palatino} % Uncomment to use the Palatino font

\usepackage{graphicx} % Required for including images
\graphicspath{{figures/}} % Location of the graphics files
\usepackage{booktabs} % Top and bottom rules for table
\usepackage[font=small,labelfont=bf]{caption} % Required for specifying captions to tables and figures
\usepackage{amsfonts, amsmath, amsthm, amssymb} % For math fonts, symbols and environments
\usepackage{wrapfig} % Allows wrapping text around tables and figures

\begin{document}

%----------------------------------------------------------------------------------------
%	POSTER HEADER 
%----------------------------------------------------------------------------------------

% The header is divided into three boxes:
% The first is 55% wide and houses the title, subtitle, names and university/organization
% The second is 25% wide and houses contact information
% The third is 19% wide and houses a logo for your university/organization or a photo of you
% The widths of these boxes can be easily edited to accommodate your content as you see fit

\begin{minipage}[b]{1\linewidth}

\center \veryHuge \color{NavyBlue} \textbf{Vertical and Oblique Transmission under Fluctuating Selection} \color{Black}\\ % Title

\vspace{1cm}

\huge \textbf{Yoav Ram$^{1}$, Uri Liberman$^{2}$, and Marcus W. Feldman$^1$}\\ % Author(s)

\vspace{1cm}

\huge $^1$ Department of Biology, Stanford University\\ % University/organization
\huge $^{2}$ School of Mathematical Sciences, Tel-Aviv University\\ % University/organization
%\huge $^{a}$yoav@yoavram.com\\ % Email

%\vspace{1cm}

\Large YR is a postdoctoral fellow of the Stanford Center for Computational, Evolutionary and Human Genomics\\ 
\faicon{envelope-o}\; yoav@yoavram.com \quad
\faicon{phone}\; 650.440.9625 \quad
\faicon{globe}\; www.yoavram.com

\end{minipage}
%
%\begin{minipage}[b]{0.25\linewidth}
%\color{DarkSlateGray}\Large \textbf{Contact Information:}\\
%Phone: +1 (650) 440-9625\\ % Phone number
%Email: \texttt{yoav@yoavram.com}\\ % Email address
%Twitter: \texttt{@yoavram}\\ % Twitter handle
%\end{minipage}
%%
%\begin{minipage}[b]{0.2\linewidth}
%\includegraphics[width=0.5\linewidth]{avatar_150x225} % Logo or a photo of you, adjust its dimensions here
%\end{minipage}

\vspace{2cm} % A bit of extra whitespace between the header and poster content

%----------------------------------------------------------------------------------------

\begin{multicols}{4} % This is how many columns your poster will be broken into, a poster with many figures may benefit from less columns whereas a text-heavy poster benefits from more

%----------------------------------------------------------------------------------------
%	ABSTRACT
%----------------------------------------------------------------------------------------

%\color{Navy} % Navy color for the abstract
%
%\begin{abstract}
%
%Sed fringilla tempus hendrerit. Vestibulum ante ipsum primis in faucibus orci luctus et ultrices posuere cubilia Curae; Etiam ut elit sit amet metus lobortis consequat sit amet in libero. Lorem ipsum dolor sit amet, consectetur adipiscing elit. Phasellus vel sem magna. Nunc at convallis urna. isus ante. Pellentesque condimentum dui. Etiam sagittis purus non tellus tempor volutpat. Donec et dui non massa tristique adipiscing. Quisque vestibulum eros eu. Phasellus imperdiet, tortor vitae congue bibendum, felis enim sagittis lorem, et volutpat ante orci sagittis mi. Morbi rutrum laoreet semper. Morbi accumsan enim nec tortor consectetur non commodo nisi sollicitudin. Proin sollicitudin. Pellentesque eget orci eros. Fusce ultricies, tellus et pellentesque fringilla, ante massa luctus libero, quis tristique purus urna nec nibh.
%
%\end{abstract}

% INTRODUCTION

\color{DarkSlateGray}

\section*{Introduction}

Oblique transmission occurs when a trait is passed to offspring from non-parental adult of the parental generation.

Examples include social learning in humans, birds, and mammals, pathogen and symbiont transmission, and HGT in microbes.

% MODEL

\section*{Model}

We model an infinite population with two phenotype, $A$ and $B$, with frequencies $x$ and $1-x$.
An offspring inherits its phenotype from its parent with probability $\rho$ or from a random adult with probability $1-\rho$.

We consider an environmental regime $AkBl$ which favors phenotype $A$ for $k$ generations and then phenotype $B$ for $l$ generations.
Let $W$ be the fitness of the favored phenotype and $w$ be that of the other phenotype where $0<w<W$.

The frequency of $A$ in the next generation $x'$, when $A$ is \mbox{favored} over $B$, is given by
\begin{equation}
x' = \rho \frac{W}{\bar{w}}x + (1-\rho) x
\end{equation}
where $\bar{w}=xW + (1-x)w$ is the population mean fitness. 


%\section*{Constant environment}
%
%When the environment is constant, the fitness parameters $w_A$ and $w_B$ do not change between generations, and we have the following result.
%
%\paragraph*{Result 1.} \textit{If $0<\rho\le 1$ and both $w_A$ and $w_B$ are positive with $w_A\ne w_B$, then fixation in the phenotype $A\ (B)$ is globally stable when $w_A>w_B$ $(w_A<w_B)$.}

\section*{Phenotype dynamics: \emph{AkBl}}

\paragraph*{Result 1.}\textit{Fixation of both $A$ and $B$ is unstable if
\begin{equation}
\frac{\log(1+\rho \frac{W-w}{w})}{\log(1+\rho \frac{w-W}{W})} < 
\frac{l}{k} <
\frac{\log(1+\rho \frac{w-W}{W})}{\log(1+\rho \frac{W-w}{w})}.
\end{equation} 
Specifically, if $k=l$ then there is a protected polymorphism.}

\begin{center}\vspace{1cm}
\includegraphics[width=0.8\linewidth]{lk_phase_plane_F}
\captionof{figure}{\textbf{Oblique transmission maintains polymorphism.}}
\end{center}\vspace{1cm}

\columnbreak

\section*{Phenotype dynamics: \emph{A1B1}}

\paragraph*{Result 2.}\textit{In selection regime $A1B1$ there exists a unique stable polymorphism:
\begin{equation}
x^* =\frac{1}{2} -\frac{W+w-\sqrt{\left(1-\rho\right)^2\left(W-w\right)^2 +4Ww}}{2\cdot(2-\rho)(W-w)},
\end{equation}
where $x^*$ is the stable frequency of phenotype $A$ after the end of each two-generation cycle.
}

Since $\bar{w}^*$ is a decreasing function function of $x^*$, the stable population mean fitness $\bar{w}^*$ is a decreasing function of $\rho$.

\begin{center}\vspace{1cm}
\includegraphics[width=0.8\linewidth]{A1B1_equilibrium_B}
\captionof{figure}{\textbf{Mean fitness in A1B1 selection regime}.
The stable population mean fitness $\bar{w}^*$ decreases as the vertical transmission rate $\rho$ increases.
}
\end{center}\vspace{1cm}

\section*{Modifier model}

Consider a modifier locus with alleles $m/M$ that determine the vertical transmission rates $\rho/P$:
\begin{equation}
\begin{matrix}
 \text{ pheno-genotype}&mA&mB&MA&MB\\
 \text{ frequency}&x_1&x_2&x_3&x_4\\
 \text{ fitness when $A$ is favored}&W&w&W&w\\
 \text{ fitness when $B$ is favored}&w&W&w&W\\
 \text{ vertical transmission rate}&\rho&\rho&P&P\\
\end{matrix}
\end{equation}

The next generation pheno-genotype frequencies when $A$ is favored are
\begin{equation}
\begin{aligned}
\hat{w}x'_1 &= W x_1\bigl[(1-\rho)(x_1+x_3) +\rho\bigr] + w x_2(1-\rho)(x_1+x_3)\\
\hat{w}x'_2 &= W x_1(1-\rho)(x_2+x_4) + w x_2\bigl[(1-\rho)(x_2+x_4)+\rho\bigr]\\ 
\hat{w}x'_3 &= W x_3\bigl[(1-P)(x_1+x_3) +P\bigr] + w x_4(1-P)(x_1+x_3)\\
\hat{w}x'_4 &= W x_3(1-P)(x_2+x_4) + w x_4\bigl[(1-P)(x_2+x_4)+P\bigr], 
\end{aligned}
\end{equation}

\paragraph{Evolutionary stable vertical transmission rate.}
We start with modifier allele $m$ with rate $\rho$ at equilibrium between phenotypes $A$ and $B$, and then introduce modifier allele $M$ with rate $P$ at a low frequency.
If $m$ is stable to invasion of any $M$ with any rate $P$, then we declare $\rho$ to be the stable rate $\rho^*$.

\columnbreak

\section*{Modifier dynamics: \emph{A1B1}}

\paragraph*{Result 3.}\textit{In selection regime $A1B1$, a modifier allele $m$ with associated vertical transmission rate $\rho$ is stable to the introduction of a modifier allele $M$ with associated rate $P$ if $P>\rho$, and it is unstable if $P<\rho$.
}

That is, there is a \emph{reduction principle} for the vertical transmission rate in selection regime $A1B1$.

Similar dynamics were observed in environments which randomly flip between favoring $A$ and $B$ every generation.


\begin{center}\vspace{1cm}
\includegraphics[width=0.8\linewidth]{A1B1_modifier_invasions}
\captionof{figure}{\textbf{Modifier invasion in A1B1 selection regime}.
Modifier alleles that reduce the vertical transmission rate successfully invade and fix in the population.
Dashed lines show invasion times. Initial resident allele with $\rho=0.1$.
}
\end{center}\vspace{1cm}

\columnbreak

\section*{Modifier dynamics: \emph{AkBk}}

In the general case there is no reduction of the transmission rate, nor is there maximization of the population mean fitness.

\begin{center}\vspace{1cm}
\includegraphics[width=0.8\linewidth]{AkBk_stable_optimal_rate}
\captionof{figure}{\textbf{Stability and optimality in selection regime $AkBk$.} \textbf{(A)} The transmission rate $\hat{\rho}$ that maximizes the geometric average of the population mean fitness over $k+l$ generations, and \textbf{(B)} the evolutionary stable vertical transmission rate $\rho^*$.
$\rho^*$ and $\hat{\rho}$ are equal to 0 and to each other only when $k=l=1$ and $k=l=2$.
}
\end{center}\vspace{1cm}

\section*{Modifier dynamics: \emph{AkBl}}

For $k \ne l$, there is a polymorphism between $A$ and $B$ when $w=0.1$ (but not when $w=0.5$ or $0.9$), in which case the stable and optimal rates are given below.

\begin{center}\vspace{1cm}
\begin{tabular}{p{2.5cm} p{2.5cm} p{4.5cm} p{4.5cm}}
\toprule
\textbf{$k$} & \textbf{$l$ } & \textbf{$\hat{\rho}$} & \textbf{$\rho^*$} \\
\midrule
1 & 2 & 0.00065 & 0.821 \\
3 & 10 & 0.00031 & 0.619 \\
5 & 30 & 0.00027 & 0.6503 \\
\bottomrule
\end{tabular}
\captionof{table}{\textbf{Stability and optimality in selection regime AkBl.}
Values of evolutionary stable vertical transmission rate $\rho^*$ and the transmission rate $\hat\rho$ that maximizes geometric mean fitness.}
\label{table}
\end{center}%\vspace{1cm}

% Conclusions

\section*{Conclusions}

\begin{itemize}
\item Oblique transmission maintains polymorphism in constant and periodically changing environments.
\item Oblique transmission is favored by selection in rapidly changing environments.
\item Vertical transmission is favored by selection in slowly changing and constant environments.
\end{itemize}

%\color{DarkSlateGray} % Set the color back to DarkSlateGray for the rest of the content

% REFERENCES

%\nocite{*} % Print all references regardless of whether they were cited in the poster or not
%\bibliographystyle{plain} % Plain referencing style
%\bibliography{sample} % Use the example bibliography file sample.bib

%----------------------------------------------------------------------------------------

\end{multicols}
\end{document}