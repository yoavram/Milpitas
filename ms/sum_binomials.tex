\documentclass[11pt, oneside]{article}   	% use "amsart" instead of "article" for AMSLaTeX format
\usepackage{geometry}                		% See geometry.pdf to learn the layout options. There are lots.
\geometry{letterpaper}                   		% ... or a4paper or a5paper or ... 
%\geometry{landscape}                		% Activate for rotated page geometry
%\usepackage[parfill]{parskip}    		% Activate to begin paragraphs with an empty line rather than an indent
\usepackage{graphicx}				% Use pdf, png, jpg, or eps§ with pdflatex; use eps in DVI mode
								% TeX will automatically convert eps --> pdf in pdflatex		
\usepackage{amsmath}
\usepackage{amssymb}

%SetFonts

%SetFonts


\begin{document}

\section*{Finite population model}
\textbf{\today}

Consider a haploid population with non-overlapping generations of constant finite size $N$.
We focus on a trait with two phenotypes: $A$ and $B$, with fitness values $w_A$ and $w_B$.
Each individual inherits his phenotype from his parent with probability $\rho$ or from a random individual from the parental generations with probability $1-\rho$.
Therefore, the number of individuals with vertically transmitted phenotypes in the next generation is:
\begin{equation}
v \sim Bin(N, \rho),
\end{equation}
and the number of individuals with diagonally transmitted phenotypes is $N-v$.

Of the individuals with vertically transmitted phenotypes, the number of individuals with phenotype $A$ depends on the parent's fitness, the frequency of $A$, and on random sampling, and is 
\begin{equation}
a_V \sim Bin(v, \frac{w_A}{\bar{w}} x),
\end{equation}
where $x$ is the frequency of phenotype $A$ in the parent generation and $\bar{w}$ is the mean fitness in the parent generation
\begin{equation}
\bar{w} = x w_A + (1-x) w_B.
\end{equation}

Of the individuals with diagonally transmitted phenotypes, the number of individuals with phenotype $A$ depends only on the frequency of $A$ and random sampling, and is 
\begin{equation}
a_D \sim Bin(N-v, x).
\end{equation}

These $a_V$ and $a_D$ are inherently separate random events, as $a_V$ depends on events that determine the reproductive success of individuals in the parental generations, whereas $a_D$ does not.

Therefore, the number of individuals with phenotype $A$ in the next generation $X_{t+1}$ given their frequency in the current generation is $x = X_t/N$, is
\begin{equation}
(X_{t+1}|X_t=Nx) = a_V + a_D.
\end{equation}
However, $(X_{t+1}|X_t=Nx)$ is not a binomial, as \textit{the sum of two binomials with different success probabilities is not a binomial}, and is underdispersed compared to a binomial (Nedelman and Wallenius, 1986):
\begin{equation}
Var(X_{t+1}|X_t=Nx) \le Var(a_V) + Var(a_D),
\end{equation}
where $a$ is a binomial only if the equality is obtained, which occurs for $\rho=0$ and $1$ or for $w_A=w_B$.

Indeed, according to eq.~2 in Nedelman and Wallenius (1986), 
\begin{equation}
Var(X_{t+1}|X_t=Nx) = 
N \Big(\bar{p} (1-\bar{p}) - \rho\Big(\frac{w_A}{\bar{w}} x - \bar{p}\Big)^2 + (1-\rho)(\bar{p} - x)^2\Big),
\end{equation}
where
\begin{equation}
\bar{p} = \rho \frac{w_A}{\bar{w}} x + (1-\rho) x.
\end{equation}

Substituting $\frac{w_A}{\bar{w}} \to 1+\frac{s}{N}(1-x)$ (see eq.~61 in main text), we have
\begin{align}
Var(X_{t+1}|X_t=Nx) = 
\frac{1}{N} x (1-x) (N^2 + N \rho s (1-2x) - \rho s^2 x (1-x)).
\end{align}

So, the variance in the frequency of $A$ in the next generation $Var\Big(\frac{X_{t+1}}{N}\Big)$ given the frequency $x$ in the current generation $t$ is
\begin{multline}
Var\Big(\frac{X_{t+1}}{N} \Big| \frac{X_{t}}{N} = x\Big) = 
\frac{1}{N^2}Var(X_{t+1}|X_t=Nx) = \\
\frac{1}{N} x (1-x) \Big(1 + \frac{1}{N} \rho s (1-2x) - \rho \frac{s^2}{N^2} x (1-x)\Big).
\end{multline}


Therefore, the diffusion approximation variance term $\sigma^2(x)$, up to terms of the order $1/N$, is
\begin{equation}
\sigma^2(x) = x(1-x),
\end{equation}
which is the same as the value given in eq.~66 of the main text if $\rho>0$ and $0<x<1$. 

\section*{References}

1.  Nedelman, J and Wallenius, T., 1986. Bernoulli trials, Poisson trials, surprising variances, and Jensen’s Inequality. The American Statistician, 40(4):286–289.


\end{document}  