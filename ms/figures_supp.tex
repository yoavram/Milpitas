\documentclass[]{article}
\usepackage{lmodern}
\usepackage{amssymb,amsmath}
\usepackage{ifxetex,ifluatex}
\usepackage{fixltx2e} % provides \textsubscript
\ifnum 0\ifxetex 1\fi\ifluatex 1\fi=0 % if pdftex
  \usepackage[T1]{fontenc}
  \usepackage[utf8]{inputenc}
\else % if luatex or xelatex
  \ifxetex
    \usepackage{mathspec}
  \else
    \usepackage{fontspec}
  \fi
  \defaultfontfeatures{Ligatures=TeX,Scale=MatchLowercase}
\fi
% use upquote if available, for straight quotes in verbatim environments
\IfFileExists{upquote.sty}{\usepackage{upquote}}{}
% use microtype if available
\IfFileExists{microtype.sty}{%
\usepackage[]{microtype}
\UseMicrotypeSet[protrusion]{basicmath} % disable protrusion for tt fonts
}{}
\PassOptionsToPackage{hyphens}{url} % url is loaded by hyperref
\usepackage[unicode=true]{hyperref}
\urlstyle{same}  % don't use monospace font for urls
\usepackage{longtable,booktabs}
% Fix footnotes in tables (requires footnote package)
\IfFileExists{footnote.sty}{\usepackage{footnote}\makesavenoteenv{long table}}{}
\usepackage{graphicx,grffile}
\makeatletter
\def\maxwidth{\ifdim\Gin@nat@width>\linewidth\linewidth\else\Gin@nat@width\fi}
\def\maxheight{\ifdim\Gin@nat@height>\textheight\textheight\else\Gin@nat@height\fi}
\makeatother
% Scale images if necessary, so that they will not overflow the page
% margins by default, and it is still possible to overwrite the defaults
% using explicit options in \includegraphics[width, height, ...]{}
\setkeys{Gin}{width=\maxwidth,height=\maxheight,keepaspectratio}
\IfFileExists{parskip.sty}{%
\usepackage{parskip}
}{% else
\setlength{\parindent}{0pt}
\setlength{\parskip}{6pt plus 2pt minus 1pt}
}
\setlength{\emergencystretch}{3em}  % prevent overfull lines
\providecommand{\tightlist}{%
  \setlength{\itemsep}{0pt}\setlength{\parskip}{0pt}}
\setcounter{secnumdepth}{0}
% Redefines (sub)paragraphs to behave more like sections
\ifx\paragraph\undefined\else
\let\oldparagraph\paragraph
\renewcommand{\paragraph}[1]{\oldparagraph{#1}\mbox{}}
\fi
\ifx\subparagraph\undefined\else
\let\oldsubparagraph\subparagraph
\renewcommand{\subparagraph}[1]{\oldsubparagraph{#1}\mbox{}}
\fi

\newcommand{\beginsupplement}{%
	\setcounter{table}{0}
	\renewcommand{\thetable}{S\arabic{table}}%
	\setcounter{figure}{0}
	\renewcommand{\thefigure}{S\arabic{figure}}%
}


% set default figure placement to htbp
\makeatletter
\def\fps@figure{htbp}
\@fpsep\textheight
\makeatother

\usepackage{lineno}
\linenumbers

\begin{document}
\pagenumbering{gobble}
\beginsupplement

\begin{figure}
\centering
\includegraphics{../figures/env_A1B1.pdf}
\caption{Frequency of phenotype \emph{A} after every two generations in
selection regime \(A1B1\). The orange line is the finite population model
(eqs. 52-53; average of 100 simulations). The blue line is the infinite population model
(eq.\ 21), and the green line is the  solution of \(Q(x)=0\) (eq.\ 22). In all cases, \(W=1\); for the finite population model (orange lines), population size is \(N=10,000\) and initial frequency of \(A\) is \(x_0=0.5\).}\label{fig:env_A1B1}
\end{figure}

\begin{figure}
\centering
\includegraphics{../figures/env_A1B2.pdf}
\caption{Frequency of phenotype \emph{A} after every three generations in
selection regime $A1B2$. Comparison of dynamics starting with
different initial frequency of phenotype \(A\) (0.01--0.99).
See also Figures 1 and  7. In all cases, $W=1$.}\label{fig:env_A1B2}
\end{figure}

\begin{figure}
\centering
\includegraphics{../figures/AkBk_x0.pdf}
\caption{Convergence of the frequency of phenotype \emph{A} to a stable polymorphism in selection regime \emph{AkBk}.
Comparison of dynamics starting with different initial frequencies of phenotype \(A\) (0.01--0.99), and different $k$, $\rho$ and $w$ values.
The lines show the $x$ frequency of phenotype \emph{A} at the end of each period, after every $2k$ generations.
In all cases, $W=1$.}\label{fig:AkBk_x0}
\end{figure}

\begin{figure}
\centering
\includegraphics{../figures/env_A3B10.pdf}
\caption{Frequency of phenotype \emph{A} after every thirteen generations in
selection regime $A3B10$. Comparison of dynamics starting with
different initial frequency of phenotype \(A\) (0.01--0.99).
In all cases, $W=1$.}\label{fig:env_A3B10}
\end{figure}

\begin{figure}
\centering
\includegraphics{../figures/AkBk_stable_wbar.pdf}
\caption{Stable population mean fitness in selection regime \emph{AkBk} as a function of the vertical transmission rate \(\rho\) and the number \(k\) of generations in which phenotypes \emph{A} and \emph{B} are favored by selection, for different selection intensities: \textbf{(A)} \(w=0.1\), \textbf{(B)} \(w=0.5\), and \textbf{(C)} \(w=0.9\).
Colors represent the geometric average of the stable population mean fitness over \(2k\) generations, calculated by iterating eq. 10 until phenotype frequencies stabilized, and for at least \(1,000 \) generations.
Blue markers show the maximum average mean fitness for each period \(k\).
For example, with \(w=0.1\), \(\hat{\rho}=0\) maximizes the average fitness for \(k \le 11\), then \(\hat{\rho}\) increases to \(\hat{\rho} \approx 0.24\), and then continues to decrease as \(k\) increases, down to \(\hat{\rho} \approx 0.15\) for \(k=50\) (see also Fig. 8A).
Contour lines represent \(\rho\) and \(k\) combinations that produce the same average mean fitness. 
In all cases, $W=1$.}\label{fig:AkBk_stable_wbar}
\end{figure}

\begin{figure}
\centering
\includegraphics{../figures/AkBk_w0.5_geomwbar.pdf}
\caption{The geometric average of the stable population mean fitness over the $2k$ generation period peaks at $\rho=0$ for $k \le 30$ (red, blue and green lines) and at $\rho \approx 0.23$ for $k=31$ and $32$ (purple and orange lines). 
See also Figs. 8A, S5B. The inset zooms out to show that the geometric mean fitness is strictly and significantly decreasing for $\rho>0.3$ (reaching $\approx 0.7$ for $\rho=1$).
In all cases, $W=1$, $w=0.5$.}\label{fig:AkBk_w0.5_geomwbar}
\end{figure}

\begin{figure}
\centering
\includegraphics{../figures/AkBk_stable_modifier_w_0.5.pdf}
 \caption{The leading eigenvalue of the external stability matrix in $AkBk$ with selection. The colors show $\lambda_1(\rho, P)$, the leading eigenvalue of the external stability matrix for combinations of $\rho$ (y-axis) and $P$ (x-axis), the vertical transmission rates of resident and invader modifiers, respectively (see \emph{Supplemental Material SP6} for details). Purple shows values lower than 1, leading to stability of $\rho$; white equals to 1; green larger than 1, leading to invasion of $P$.
  The arrows show an "evolutionary" path $\rho_{0} \to \rho_{1} \to \ldots \to \rho^*$: at each step, a modifier with rate $P_t$ invades the resident modifier with rate $\rho_t$ and becomes the resident modifier $\rho_{t+1}$, until consecutive invaders have similar rates.
  Here, $W=1$ and $w=0.5$.}\label{fig:AkBk_stable_modifier_w_0.5}
\end{figure}

\end{document}  