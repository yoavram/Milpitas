

\begin{figure}
\centering
\includegraphics{figures/lk_phase_plane.pdf}
\caption{Ratios of selection periods \(\frac{l}{k}\) that lead to
fixation of phenotype \emph{B} (red) or polymorphism of phenotypes
\emph{A} and \emph{B} (blue). \emph{l} and \emph{k} are the number of
generations in which phenotypes \emph{B} and \emph{A}, respectively, are favored by
selection. In all cases, \emph{W} = 1.}\label{fig:lk_phase_plane}
\end{figure}

\begin{figure}
\centering
\includegraphics{figures/env_A1B1.pdf}
\caption{Frequency of phenotype \emph{A} after every two generations in
selection regime \emph{A1B1}. The right column is the finite population model
(orange; average of 100 simulations). The left column is the infinite population model
(blue; @eq:recurrenceA1B1), and the equilibrium solution (dashed green;
@eq:recurrenceA1B1\_solution\_x\_star). Parameters: \emph{W}=1,
\emph{N}=10,000, initial value \(x=0.5\).}\label{fig:env_A1B1}
\end{figure}

\begin{figure}
\centering
\includegraphics{figures/A1B1_modifier_invasions.pdf}
\caption{Consecutive fixation of modifiers that reduce the vertical
transmission rate in environmental regime \emph{A1B1}. The figure shows
results of numerical simulations of evolution with two modifier alleles
(@eq:recurrence\_modifiers). When a modifier allele fixes
(frequency\textgreater{}99.9\%), a new modifier allele is introduced
with a vertical transmission rate one order of magnitude lower (vertical
dashed lines). \textbf{(A,E,I)} The frequency of phenotype \emph{A} in
the population over time. \textbf{(B,F,J)} The frequency of the invading
modifier allele over time. \textbf{(C,G,K)} The population mean vertical
transmission rate over time. \textbf{(D, H, L)} The population mean
fitness over time; insets zoom in to show that the mean fitness 
decreases slightly with each invasion. Parameters: vertical transmission rate of
the initial resident modifier allele, \(\rho_0 =0.1\); fitness values:
\emph{W}=1, \emph{w}=0.1 (\textbf{A-D}), 0.5 (\textbf{E-H}), and 0.9
(\textbf{I-L}). The x-axis is on a log-scale, as each sequential invasion
takes an order of magnitude longer to
complete.}\label{fig:A1B1_modifier_invasions}
\end{figure}

\begin{figure}
\centering
\includegraphics{figures/A1B1_stable_x.pdf}
\caption{Stable frequency of phenotype \emph{A}
(@eq:recurrenceA1B1\_solution\_x\_star) in selection regime \emph{A1B1}
as a function of the vertical transmission rate \(\rho\) and the fitness
ratio \(W/w\) between the favored and unfavored phenotypes. Here
\emph{W=1}.}\label{fig:A1B1_stable_x}
\end{figure}

\begin{figure}
\centering
\includegraphics{figures/A1B1_mean_fitness.pdf}
\caption{Stable mean fitness in selection regime \emph{A1B1} as a
function of the vertical transmission rate \(\rho\) and the fitness
ratio \(W/w\) between the favored and unfavored phenotypes. Here 
\emph{W=1}.}\label{fig:A1B1_mean_fitness}
\end{figure}

\begin{figure}
\centering
\includegraphics{figures/env_A1B2.pdf}
\caption{Frequency of phenotype \emph{A} after every three generation in
selection regime \emph{A1B2}. Comparison of dynamics starting with
different initial frequency of phenotype \(A\) (0.01-0.99). Fixation occurs if the final population in 100 simulations with finite
population size \(N\) has zero frequency of \(A\) at the end of the run. See also @Fig:lk\_phase\_plane
and @Fig:A1B2\_polymorphism\_x. Parameters: \emph{W}=1, \emph{N=10,000},
initial population \(x=0.5\).}\label{fig:env_A1B2}
\end{figure}

\begin{figure}
\centering
\includegraphics{figures/A1B2_polymorphism_x.pdf}
\caption{Stable frequency of phenotype \emph{A} in selection regime
\emph{A1B2} as a function of the vertical transmission rate \(\rho\) and
the fitness ratio \(W/w\) between the favored and unfavored phenotypes.
In the grey area \(B\) reached fixation and the frequency of \(A\) is 0.
Here \emph{W=1}.}\label{fig:A1B2_polymorphism_x}
\end{figure}

\begin{figure}
\centering
\includegraphics{figures/stochastic_env_x_t.pdf}
\caption{Stochastic local stability. The figure shows the frequency of
phenotype \emph{A} after \(10^6\) generations in a very large population
evolving in a stochastic environment (@eq:recurrence\_random\_env). The
fitness of phenotypes \emph{A} and \emph{B} are \(1+s_t\) and \(1\),
where \(s_t\) is \emph{s} with probability \emph{p} and \emph{-s} with
probability \emph{1-p}. The white line marks combinations of \emph{p}
and \emph{s} for which \(\mathbb{E}[\log{(1+\rho s_t)}]=0\); according
to our analysis, we expect that below this line \(x^*=0\) will be
stochastically locally stable. Parameters: \(x_0=0.1\);
\(\rho=0.1\).}\label{fig:stochastic_env_x_t}
\end{figure}

\begin{figure}
\centering
\includegraphics{figures/stoch_modifier_invasions.png}
\caption{Consecutive fixation of modifiers that reduce the vertical
transmission rate under randomly changing selection. The figure shows
results of numerical simulations of evolution with two modifier alleles
(@eq:recurrence\_modifiers). When a modifier allele fixes
(frequency\textgreater{}99.9\%), a new modifier allele is introduced
with a vertical transmission rate one order of magnitude lower (vertical
dashed lines). \textbf{(A)} The frequency of phenotype \emph{A} in the
population over time. \textbf{(B)} The frequency of the invading
modifier allele over time. \textbf{(C)} The population mean vertical
transmission rate over time. Parameters: vertical transmission rate of the initial
resident modifier allele, \(\rho_0 =0.1\); fitness values: \emph{W=1}
and \emph{w=0.1} with probability 0.5 and \emph{w=0.1} and \emph{W=1}
otherwise. The x-axis is on a log-scale, as each sequential invasion takes
an order of magnitude longer to
complete.}\label{fig:stoch_modifier_invasions}
\end{figure}

\begin{figure}
\centering
\includegraphics{figures/stoch_two_iid_rvs.png}
\caption{Effect of vertical transmission rate on phenotype polymorphism
in randomly changing environments. Dynamics of the frequency of phenotype
\(A\) over time starting
at \(x_0=10^{-5}\) when the fitness of phenotypes \(A\) and \(B\) are
identically and independently distributed random variable. As the vertical transmission rate \(\rho\) increases
from 0.001 to 0.5 the frequency reaches a polymorphic distribution with \(E(X_t/N)\to0.5\) faster, but
the variance also increases. \(w_A\) and \(w_B\) are both exponential
random variables with expected value 2.}\label{fig:stoch_two_iid_rvs}
\end{figure}

\begin{figure}
\centering
\includegraphics{figures/fixation_prob_time.pdf}
\caption{Fixation probability and time in a finite population.
\textbf{(A)} Fixation probability \(u(x)\) of phenotype \(A\) and \textbf{(B)}
Expected time to fixation \(T(x)\) of phenotype \(A\), conditioned on its
fixation, starting with a single copy in a population of size \(N\). The
figure compares two estimates: simulations (blue circles) and diffusion
equation approximation (green solid line). Parameters: Selection coefficient, \(s=w_A-w_B=0.1\);
Population size, \(N=10,000\).}\label{fixation_prob_time}
\end{figure}
