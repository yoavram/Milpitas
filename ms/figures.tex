\documentclass[]{article}
\usepackage{lmodern}
\usepackage{amssymb,amsmath}
\usepackage{ifxetex,ifluatex}
\usepackage{fixltx2e} % provides \textsubscript
\ifnum 0\ifxetex 1\fi\ifluatex 1\fi=0 % if pdftex
  \usepackage[T1]{fontenc}
  \usepackage[utf8]{inputenc}
\else % if luatex or xelatex
  \ifxetex
    \usepackage{mathspec}
  \else
    \usepackage{fontspec}
  \fi
  \defaultfontfeatures{Ligatures=TeX,Scale=MatchLowercase}
\fi
% use upquote if available, for straight quotes in verbatim environments
\IfFileExists{upquote.sty}{\usepackage{upquote}}{}
% use microtype if available
\IfFileExists{microtype.sty}{%
\usepackage[]{microtype}
\UseMicrotypeSet[protrusion]{basicmath} % disable protrusion for tt fonts
}{}
\PassOptionsToPackage{hyphens}{url} % url is loaded by hyperref
\usepackage[unicode=true]{hyperref}
\urlstyle{same}  % don't use monospace font for urls
\usepackage{longtable,booktabs}
% Fix footnotes in tables (requires footnote package)
\IfFileExists{footnote.sty}{\usepackage{footnote}\makesavenoteenv{long table}}{}
\usepackage{graphicx,grffile}
\makeatletter
\def\maxwidth{\ifdim\Gin@nat@width>\linewidth\linewidth\else\Gin@nat@width\fi}
\def\maxheight{\ifdim\Gin@nat@height>\textheight\textheight\else\Gin@nat@height\fi}
\makeatother
% Scale images if necessary, so that they will not overflow the page
% margins by default, and it is still possible to overwrite the defaults
% using explicit options in \includegraphics[width, height, ...]{}
\setkeys{Gin}{width=\maxwidth,height=\maxheight,keepaspectratio}
\IfFileExists{parskip.sty}{%
\usepackage{parskip}
}{% else
\setlength{\parindent}{0pt}
\setlength{\parskip}{6pt plus 2pt minus 1pt}
}
\setlength{\emergencystretch}{3em}  % prevent overfull lines
\providecommand{\tightlist}{%
  \setlength{\itemsep}{0pt}\setlength{\parskip}{0pt}}
\setcounter{secnumdepth}{0}
% Redefines (sub)paragraphs to behave more like sections
\ifx\paragraph\undefined\else
\let\oldparagraph\paragraph
\renewcommand{\paragraph}[1]{\oldparagraph{#1}\mbox{}}
\fi
\ifx\subparagraph\undefined\else
\let\oldsubparagraph\subparagraph
\renewcommand{\subparagraph}[1]{\oldsubparagraph{#1}\mbox{}}
\fi

% set default figure placement to htbp
\makeatletter
\def\fps@figure{htbp}
\makeatother

\usepackage{lineno}
\linenumbers

\begin{document}
\pagenumbering{gobble} % remove page numbers

%Figure 1
\begin{figure}
\centering
\includegraphics{../figures/lk_phase_plane.pdf}
\caption{Ratios of selection periods \(\frac{k}{l}\) that lead to
fixation of phenotype \emph{A} (red) or polymorphism of phenotypes
\emph{A} and \emph{B} (blue). \emph{k} and \emph{l} are the number of
generations in which phenotypes \emph{A} and \emph{B}, respectively, are favored by selection.
In all cases, \emph{W} = 1, \emph{w} = \(1-s\).}\label{fig:lk_phase_plane}
\end{figure}

%Figure 2
\begin{figure}
\centering
\includegraphics{../figures/A1B1_equilibrium.pdf}
\caption{Properties of stability in \emph{A1B1} selection regime.
\textbf{(A)} Stable frequency of phenotype $A$ and \textbf{(B)} stable mean fitness as functions of the vertical transmission rate \(\rho\) and the fitness of the disfavored phenotype \(w\).
Black contour lines join \(\rho\) and \(w\) combinations that result in the same stable value.
In all cases, fitness of the favored phenotype is \emph{W=1}.
}\label{fig:A1B1_equilibrium}
\end{figure}

%Figure 3
\begin{figure}
\centering
\includegraphics{../figures/stochastic_env_x_t.pdf}
\caption{Stochastic local stability. 
The figure shows the frequency of phenotype \emph{A} after \(10^6\) generations in a very large population evolving in a stochastic environment (eq.\ 28).
The fitnesses of phenotypes \emph{A} and \emph{B} are \(w_A=1+s_t\) and \(w_B=1\),
where \(s_t\) is \emph{s} with probability \emph{p} and \emph{-s} with probability \(1-p\).
The gray lines mark combinations of \emph{p} and \emph{s} for which \(E[\log{(1-\rho+\rho\frac{w_A}{w_B})}]=0\) and \(E[\log{(1-\rho+\rho\frac{w_B}{w_A})}]=0\).
According to Result 6, between these lines fixation of either phenotype is not stochastically locally stable, and we expect a stationary polymorphism between the lines.
Here, initial frequency of \(A\) is \(x_0=1/10,000\) and the vertical transmission rate is \(\rho=0.1\).}\label{fig:stochastic_env_x_t}
\end{figure}

%Figure 4
\begin{figure}
\centering
\includegraphics{../figures/stoch_two_iid_rvs.pdf}
\caption{Effect of vertical transmission rate $\rho$ on phenotype polymorphism in a randomly changing environment. Dynamics of the frequency of phenotype \(A\) over time starting at \(x_0=10^{-5}\) when the fitnesses of phenotypes \(A\) and \(B\) are identically and independently distributed random variables.
As the vertical transmission rate \(\rho\) increases from 0.001 to 0.5, the frequency reaches a polymorphic distribution with \(E(x_t)\to0.5\) faster,
but the variance also increases.
The fitnesses of phenotypes \(A\) and \(B\), \(w_A\) and \(w_B\), are both exponential random variables with expected value 2.}\label{fig:stoch_two_iid_rvs}
\end{figure}

%Figure 5
\begin{figure}
\centering
\includegraphics{../figures/A1B1_modifier_invasions.pdf}
\caption{Consecutive fixation of modifiers that reduce the vertical
transmission rate in selection regime \emph{A1B1}. The figure shows
results of numerical simulations of evolution with two modifier alleles
(eq.\ 32). When a modifier allele fixes
(frequency\textgreater{}99.9\%), a new modifier allele is introduced
with a vertical transmission rate one order of magnitude lower (vertical
dashed lines). \textbf{(A,D,G)} The frequency of phenotype \emph{A} in
the population over time. \textbf{(B,E,H)} The frequency of the invading
modifier allele over time. \textbf{(C,F,I)} The population mean
fitness over time; insets zoom in to show that the mean fitness 
decreases slightly with each invasion. Parameters: vertical transmission rate of
the initial resident modifier allele, \(\rho_0 =0.1\); fitness values:
$W=1$; $w=0.1$ (\textbf{A-C}), 0.5 (\textbf{D-F}), and 0.9
(\textbf{G-I}). The x-axis is on a log-scale, as each sequential invasion
takes an order of magnitude longer to
complete.}\label{fig:A1B1_modifier_invasions}
\end{figure}

%Figure 6
\begin{figure}
\centering
\includegraphics[height=\dimexpr \textheight - 8\baselineskip\relax]{../figures/stoch_modifier_invasions.pdf}
\caption{Consecutive fixation of modifiers that reduce the vertical transmission rate $\rho$ under symmetric randomly changing selection.
The figure shows results of numerical simulations of evolution with two modifier alleles
(eq.\ 32).
When a modifier allele fixes (frequency\textgreater{}99.9\%),
a new modifier allele is introduced with a vertical transmission rate one order of magnitude lower (vertical dashed lines).
\textbf{(A)} The frequency of phenotype \emph{A} in the population over time.
\textbf{(B)} The frequency of the invading modifier allele over time.
Parameters: vertical transmission rate of the initial resident modifier allele is \(\rho_0 =0.1\) and the ratio of fitness values  is $w_A/w_B=10$ with probability 0.5 and $w_A/w_B=0.1$ also with probability 0.5.
The x-axis is on a log-scale, as each sequential invasion takes an order of magnitude longer to complete.}\label{fig:stoch_modifier_invasions}
\end{figure}

%Figure 7
\begin{figure}
\centering
\includegraphics{../figures/A1B2_polymorphism.pdf}
\caption{Stable frequency of phenotype \emph{A} and geometric mean fitness in selection regime
$A1B2$ as a function of the vertical transmission rate \(\rho\) and the fitness of the disfavored phenotype \(w\).
\textbf{(A)} Stable frequency of phenotype $A$ at the end of each three generations cycle.
\textbf{(B)} Geometric average of the stable population mean fitness over the three generations cycle: $(\bar{w}^* \cdot \bar{w}^{**} \cdot \bar{w}^{***})^{1/3}$.
Gray contour lines join \(\rho\) and \(w\) combinations that result in the same stable value.
In all cases, \emph{W=1}.}\label{fig:A1B2_polymorphism}
\end{figure}

%Figure 8
\begin{figure}
\centering
\includegraphics{../figures/AkBk_stable_optimal_rate.pdf}
\caption{Fitness ``optimal'' and evolutionary stable vertical transmission rate in $AkBk$ selection regime. 
\textbf{(A)} The vertical transmission rate $\hat{\rho}$ that maximized the geometric average of the population mean fitness is zero (complete oblique transmission) when selection cycles quickly between favoring phenotype $A$ and $B$, and then abruptly transitions to $\approx 0.2$, followed by a slow decrease (see Fig. S5 and S6 for details on the abrupt transition).
\textbf{(B)} The evolutionary stable rate $\rho^*$, which cannot be invaded by modifiers with either higher or lower vertical transmission rate $P$, rapidly increases from zero when selection cycles are short ($k=1$ or $2$) to roughly 1 when selection cycles are longer.
The dashed line shows $1-\frac{1}{k-1}$, which fit the values for $w=0.5$ (Carja et al. 2011). The values for $w=0.1$ (blue) could not be calculated for $k > 19$ due to numerical instability when selection is strong and the duration between selection fluctuations is long.
In all cases, $W=1$. See \emph{Supplemental Material SP6} for details on how we calculated the stable rate.}\label{fig:AkBk_stable_optimal_rate}
\end{figure}

%Figure 9
\begin{figure}
\centering
\includegraphics{../figures/fixation_prob_time.pdf}
\caption{Fixation probability and time in a finite population.
\textbf{(A)} Fixation probability \(u(x)\) of phenotype \(A\) (eq.\ 55), and \textbf{(B)}
Expected time to fixation \(T(x)\) of phenotype \(A\) (eq.\ 56) conditioned on its
fixation, starting with a single copy in a population of size \(N\). The
figure compares two estimates: Wright-Fisher simulations (blue circles) and diffusion
equation approximation (green solid line). Parameters: selection coefficient, \(s=w_A-w_B=0.1\),
population size, \(N=10,000\).}\label{fixation_prob_time}
\end{figure}

%Figure 10
\begin{figure}
\centering
\includegraphics{../figures/lk_fix_prob.pdf}
\caption{Fixation in a finite population with different ratios of selection periods \(\frac{k}{l}\). Fixation probability of phenotype $A$ when starting with a single copy in a population of size $N$: $u(1/N) = (1-\exp(-2 \rho \frac{k-l}{k+l}(W-w))/(1-\exp(-2 N \rho \frac{k-l}{k+l}(W-w))$ (see eqs. 59--60).
\emph{k} and \emph{l} are the number of
generations in which phenotypes \emph{A} and \emph{B}, respectively, are favored by
selection. In all cases, fitness of the favored phenotype, $W = 1$; fitness of the unfavored phenotype is $w=1-s$ and the population size is \(N=10,000\).} \label{lk_fix_prob}
\end{figure}

%Figure 10
\begin{figure}
\centering
\includegraphics{../figures/lk_fix_prob_C.pdf}
\caption{Fixation in a finite population with different ratios of selection periods \(\frac{k}{l}\). Fixation probability of phenotype $A$ when starting with a single copy in a population of size $N$: $u(1/N) = (1-\exp(-2 \rho \frac{k-l}{k+l}(W-w))/(1-\exp(-2 N \rho \frac{k-l}{k+l}(W-w))$ (see eqs. 59--60).
\emph{k} and \emph{l} are the number of
generations in which phenotypes \emph{A} and \emph{B}, respectively, are favored by
selection. Here, fitness of the favored phenotype is $W = 1$, fitness of the unfavored phenotype is $w=0.5$, and the population size is \(N=10,000\). 
See Figure SX for additional parameters.} \label{lk_fix_prob_C}
\end{figure}

\end{document}  