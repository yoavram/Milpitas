\documentclass[]{article}
\usepackage{lmodern}
\usepackage{amssymb,amsmath}
\usepackage{ifxetex,ifluatex}
\usepackage{fixltx2e} % provides \textsubscript
\ifnum 0\ifxetex 1\fi\ifluatex 1\fi=0 % if pdftex
  \usepackage[T1]{fontenc}
  \usepackage[utf8]{inputenc}
\else % if luatex or xelatex
  \ifxetex
    \usepackage{mathspec}
  \else
    \usepackage{fontspec}
  \fi
  \defaultfontfeatures{Ligatures=TeX,Scale=MatchLowercase}
\fi
% use upquote if available, for straight quotes in verbatim environments
\IfFileExists{upquote.sty}{\usepackage{upquote}}{}
% use microtype if available
\IfFileExists{microtype.sty}{%
\usepackage[]{microtype}
\UseMicrotypeSet[protrusion]{basicmath} % disable protrusion for tt fonts
}{}
\PassOptionsToPackage{hyphens}{url} % url is loaded by hyperref
\usepackage[unicode=true]{hyperref}
\urlstyle{same}  % don't use monospace font for urls
\usepackage{longtable,booktabs}
% Fix footnotes in tables (requires footnote package)
\IfFileExists{footnote.sty}{\usepackage{footnote}\makesavenoteenv{long table}}{}
\usepackage{graphicx,grffile}
\makeatletter
\def\maxwidth{\ifdim\Gin@nat@width>\linewidth\linewidth\else\Gin@nat@width\fi}
\def\maxheight{\ifdim\Gin@nat@height>\textheight\textheight\else\Gin@nat@height\fi}
\makeatother
% Scale images if necessary, so that they will not overflow the page
% margins by default, and it is still possible to overwrite the defaults
% using explicit options in \includegraphics[width, height, ...]{}
\setkeys{Gin}{width=\maxwidth,height=\maxheight,keepaspectratio}
\IfFileExists{parskip.sty}{%
\usepackage{parskip}
}{% else
\setlength{\parindent}{0pt}
\setlength{\parskip}{6pt plus 2pt minus 1pt}
}
\setlength{\emergencystretch}{3em}  % prevent overfull lines
\providecommand{\tightlist}{%
  \setlength{\itemsep}{0pt}\setlength{\parskip}{0pt}}
\setcounter{secnumdepth}{0}
% Redefines (sub)paragraphs to behave more like sections
\ifx\paragraph\undefined\else
\let\oldparagraph\paragraph
\renewcommand{\paragraph}[1]{\oldparagraph{#1}\mbox{}}
\fi
\ifx\subparagraph\undefined\else
\let\oldsubparagraph\subparagraph
\renewcommand{\subparagraph}[1]{\oldsubparagraph{#1}\mbox{}}
\fi

% set default figure placement to htbp
\makeatletter
\def\fps@figure{htbp}
\makeatother

\usepackage{lineno}
\linenumbers

\begin{document}

\begin{figure}
\centering
\includegraphics{../figures/lk_phase_plane.pdf}
\caption{Ratios of selection periods \(\frac{k}{l}\) that lead to
fixation of phenotype \emph{A} (red) or polymorphism of phenotypes
\emph{A} and \emph{B} (blue). \emph{k} and \emph{l} are the number of
generations in which phenotypes \emph{A} and \emph{B}, respectively, are favored by selection.
The boundary is set at $-\log{(1+\rho \frac{s}{1-s}}) / \log{(1-\rho s})$, see eq. 23.
In all cases, \emph{W} = 1, \emph{w} = \emph{1-s}.}\label{fig:lk_phase_plane}
\end{figure}

\begin{figure}
\centering
\includegraphics{../figures/env_A1B1.pdf}
\caption{Frequency of phenotype \emph{A} after every two generations in
selection regime \emph{A1B1}. The orange line is the finite population model
(average of 100 simulations). The blue line is the infinite population model
(eq. 25), and the green line is the equilibrium solution (eq. 56). Parameters: \emph{W}=1,
\emph{N}=10,000, initial value \(x=0.5\).}\label{fig:env_A1B1}
\end{figure}

\begin{figure}
\centering
\includegraphics{../figures/A1B1_modifier_invasions.pdf}
\caption{Consecutive fixation of modifiers that reduce the vertical
transmission rate in environmental regime \emph{A1B1}. The figure shows
results of numerical simulations of evolution with two modifier alleles
(eq. 53). When a modifier allele fixes
(frequency\textgreater{}99.9\%), a new modifier allele is introduced
with a vertical transmission rate one order of magnitude lower (vertical
dashed lines). \textbf{(A,D,G)} The frequency of phenotype \emph{A} in
the population over time. \textbf{(B,E,H)} The frequency of the invading
modifier allele over time. \textbf{(C,F,I)} The population mean
fitness over time; insets zoom in to show that the mean fitness 
decreases slightly with each invasion. Parameters: vertical transmission rate of
the initial resident modifier allele, \(\rho_0 =0.1\); fitness values:
\emph{W}=1, \emph{w}=0.1 (\textbf{A-C}), 0.5 (\textbf{D-F}), and 0.9
(\textbf{G-I}). The x-axis is on a log-scale, as each sequential invasion
takes an order of magnitude longer to
complete.}\label{fig:A1B1_modifier_invasions}
\end{figure}

\begin{figure}
\centering
\includegraphics{../figures/AkBl_equilibrium.pdf}
\caption{Stability in a fluctuating environment.
\textbf{(A)} Stable frequency of phenotype $A$ after every two generations and \textbf{(B)} stable mean fitness in selection regime \emph{A1B1} as a function of the vertical transmission rate \(\rho\) and the fitness ratio \(W/w\) between the favored and unfavored phenotypes.
\textbf{(C)} Stable population mean fitness in selection regime \emph{AkBl} as a
function of the vertical transmission rate \(\rho\) and different ratios of selection periods \(k=l\). \emph{k} and \emph{l} are the number of generations in which phenotypes \emph{A} and \emph{B}, respectively, are favored by selection.
The geometric average of the population mean fitness over $k+l$ generations is shown. 
Here \emph{w=0.9}.
In all cases, \emph{W=1}.
}\label{fig:AkBl_equilibrium}
\end{figure}

\begin{figure}
\centering
\includegraphics{../figures/env_A1B2.pdf}
\caption{Frequency of phenotype \emph{A} after every three generation in
selection regime \emph{A1B2}. Comparison of dynamics starting with
different initial frequency of phenotype \(A\) (0.01-0.99).
See also Fig. 1 and 6. Parameters: \emph{W}=1, \emph{N=10,000},
initial population \(x=0.5\).}\label{fig:env_A1B2}
\end{figure}

\begin{figure}
\centering
\includegraphics{../figures/A1B2_polymorphism.pdf}
\caption{Stable frequency of phenotype \emph{A} and population mean fitness in selection regime
\emph{A1B2} as a function of the vertical transmission rate \(\rho\) and the fitness ratio \(W/w\) between the favored and unfavored phenotypes.
In the grey area \(B\) reached fixation and the frequency of \(A\) is 0.
\textbf{(A)} Stable frequency of phenotype $A$ at the end of each period of three generations.
\textbf{(B)} Geometric average of the stable mean fitness over the three generations period: $(\bar{w}^* \cdot \bar{w}^{**} \cdot \bar{w}^{***})^{1/3}$.
In all cases, \emph{W=1}.}\label{fig:A1B2_polymorphism}
\end{figure}

\begin{figure}
\centering
\includegraphics{../figures/stochastic_env_x_t.pdf}
\caption{Stochastic local stability. The figure shows the frequency of
phenotype \emph{A} after \(10^6\) generations in a very large populationt 
evolving in a stochastic environment (eq. 33). The
fitness of phenotypes \emph{A} and \emph{B} are \(1+s_t\) and \(1\),
where \(s_t\) is \emph{s} with probability \emph{p} and \emph{-s} with
probability \emph{1-p}. The white line marks combinations of \emph{p}
and \emph{s} for which \(\mathbb{E}[\log{(1+\rho s_t)}]=0\); according
to our analysis, we expect that below this line \(x^*=0\) will be
stochastically locally stable. Parameters: \(x_0=0.1\);
\(\rho=0.1\).}\label{fig:stochastic_env_x_t}
\end{figure}

\begin{figure}
\centering
\includegraphics[height=\dimexpr \textheight - 8\baselineskip\relax]{../figures/stoch_modifier_invasions.png}
\caption{Consecutive fixation of modifiers that reduce the vertical
transmission rate under randomly changing selection. The figure shows
results of numerical simulations of evolution with two modifier alleles
(eq. 33). The
fitness of phenotypes \emph{A} and \emph{B} are 1 and 0.1 with probability 0.5 or 0.1 and 1 otherwise.
When a modifier allele fixes
(frequency\textgreater{}99.9\%), a new modifier allele is introduced
with a vertical transmission rate one order of magnitude lower (vertical
dashed lines). \textbf{(A)} The frequency of phenotype \emph{A} in the
population over time. \textbf{(B)} The frequency of the invading
modifier allele over time. \textbf{(C)} The population mean fitness over time. Parameters: vertical transmission rate of the initial
resident modifier allele, \(\rho_0 =0.1\); fitness values: \emph{W=1}
and \emph{w=0.1} with probability 0.5 and \emph{w=0.1} and \emph{W=1}
otherwise. The x-axis is on a log-scale, as each sequential invasion takes
an order of magnitude longer to
complete.}\label{fig:stoch_modifier_invasions}
\end{figure}

\begin{figure}
\centering
\includegraphics{../figures/stoch_two_iid_rvs.png}
\caption{Effect of vertical transmission rate on phenotype polymorphism
in randomly changing environments. Dynamics of the frequency of phenotype
\(A\) over time starting
at \(x_0=10^{-5}\) when the fitness of phenotypes \(A\) and \(B\) are
identically and independently distributed random variable. As the vertical transmission rate \(\rho\) increases
from 0.001 to 0.5 the frequency reaches a polymorphic distribution with \(E(X_t/N)\to0.5\) faster, but
the variance also increases. \(w_A\) and \(w_B\) are both exponential
random variables with expected value 2.}\label{fig:stoch_two_iid_rvs}
\end{figure}

\begin{figure}
\centering
\includegraphics{../figures/fixation_prob_time.pdf}
\caption{Fixation probability and time in a finite population.
\textbf{(A)} Fixation probability \(u(x)\) of phenotype \(A\) (eq. 88), and \textbf{(B)}
Expected time to fixation \(T(x)\) of phenotype \(A\) (eq. 90), conditioned on its
fixation, starting with a single copy in a population of size \(N\). The
figure compares two estimates: simulations (blue circles) and diffusion
equation approximation (green solid line). Parameters: Selection coefficient, \(s=w_A-w_B=0.1\);
Population size, \(N=10,000\).}\label{fixation_prob_time}
\end{figure}

\begin{figure}
\centering
\includegraphics{../figures/lk_fix_prob.pdf}
\caption{Fixation in a finite population with different ratios of selection periods \(\frac{k}{l}\). Fixation probability of phenotype $A$ when starting with a single copy in a population of size $N$: $u(1/N) = (1-\exp(-2 \rho \frac{k-l}{k+l}(W-w))/(1-\exp(-2 N \rho \frac{k-l}{k+l}(W-w))$ (see eqs. 115-116).
\emph{k} and \emph{l} are the number of
generations in which phenotypes \emph{A} and \emph{B}, respectively, are favored by
selection. In all cases, \emph{W} = 1, \emph{w} = \emph{1-s}, Population size, \(N=10,000\).} \label{lk_fix_prob}
\end{figure}

\end{document}  