\documentclass[]{article}
\usepackage{lmodern}
\usepackage{amssymb,amsmath}
\usepackage{ifxetex,ifluatex}
\usepackage{fixltx2e} % provides \textsubscript
\ifnum 0\ifxetex 1\fi\ifluatex 1\fi=0 % if pdftex
  \usepackage[T1]{fontenc}
  \usepackage[utf8]{inputenc}
\else % if luatex or xelatex
  \ifxetex
    \usepackage{mathspec}
  \else
    \usepackage{fontspec}
  \fi
  \defaultfontfeatures{Ligatures=TeX,Scale=MatchLowercase}
\fi
% use upquote if available, for straight quotes in verbatim environments
\IfFileExists{upquote.sty}{\usepackage{upquote}}{}
% use microtype if available
\IfFileExists{microtype.sty}{%
\usepackage[]{microtype}
\UseMicrotypeSet[protrusion]{basicmath} % disable protrusion for tt fonts
}{}
\PassOptionsToPackage{hyphens}{url} % url is loaded by hyperref
\usepackage[unicode=true]{hyperref}
\urlstyle{same}  % don't use monospace font for urls
\usepackage{longtable,booktabs}
% Fix footnotes in tables (requires footnote package)
\IfFileExists{footnote.sty}{\usepackage{footnote}\makesavenoteenv{long table}}{}
\usepackage{graphicx,grffile}
\makeatletter
\def\maxwidth{\ifdim\Gin@nat@width>\linewidth\linewidth\else\Gin@nat@width\fi}
\def\maxheight{\ifdim\Gin@nat@height>\textheight\textheight\else\Gin@nat@height\fi}
\makeatother
% Scale images if necessary, so that they will not overflow the page
% margins by default, and it is still possible to overwrite the defaults
% using explicit options in \includegraphics[width, height, ...]{}
\setkeys{Gin}{width=\maxwidth,height=\maxheight,keepaspectratio}
\IfFileExists{parskip.sty}{%
\usepackage{parskip}
}{% else
\setlength{\parindent}{0pt}
\setlength{\parskip}{6pt plus 2pt minus 1pt}
}
\setlength{\emergencystretch}{3em}  % prevent overfull lines
\providecommand{\tightlist}{%
  \setlength{\itemsep}{0pt}\setlength{\parskip}{0pt}}
\setcounter{secnumdepth}{0}
% Redefines (sub)paragraphs to behave more like sections
\ifx\paragraph\undefined\else
\let\oldparagraph\paragraph
\renewcommand{\paragraph}[1]{\oldparagraph{#1}\mbox{}}
\fi
\ifx\subparagraph\undefined\else
\let\oldsubparagraph\subparagraph
\renewcommand{\subparagraph}[1]{\oldsubparagraph{#1}\mbox{}}
\fi

% set default figure placement to htbp
\makeatletter
\def\fps@figure{htbp}
\makeatother

\usepackage{lineno}
\linenumbers

\begin{document}
\pagenumbering{gobble} % remove page numbers

%Figure 1
\begin{figure}
\centering
\includegraphics{../figures/lk_phase_plane.pdf}
\caption{Ratios of selection periods \(\frac{k}{l}\) that lead to
fixation of phenotype \emph{A} (red) or polymorphism of phenotypes
\emph{A} and \emph{B} (blue). \emph{k} and \emph{l} are the number of
generations in which phenotypes \emph{A} and \emph{B}, respectively, are favored by selection.
In all cases, \emph{W} = 1, \emph{w} = \(1-s\).}\label{fig:lk_phase_plane}
\end{figure}

%Figure 2
\begin{figure}
\centering
\includegraphics{../figures/env_A1B1.pdf}
\caption{Frequency of phenotype \emph{A} after every two generations in
selection regime \(A1B1\). The orange line is the finite population model
(eqs. 52-53; average of 100 simulations). The blue line is the infinite population model
(eq.\ 23), and the green line is the  solution of \(G(x)=0\) (see eq.\ 24). Parameters: \(W=1\); For the finite population model (orange lines), population size is \(N=10,000\) and initial frequency of \(A\) is \(x_0=0.5\).}\label{fig:env_A1B1}
\end{figure}

%Figure 3
\begin{figure}
\centering
\includegraphics{../figures/A1B1_equilibrium.pdf}
\caption{Properties of $A1B1$ environment.
\textbf{(A)} Stable frequency of phenotype $A$ in the \emph{A1B1} selection regime and \textbf{(B)} stable mean fitness in selection regime \emph{A1B1} as functions of the vertical transmission rate \(\rho\) and the fitness ratio \(w/W\) between the favored and unfavored phenotypes.
In all cases, \emph{W=1}.
}\label{fig:AkBl_equilibrium}
\end{figure}

%Figure 4
\begin{figure}
\centering
\includegraphics{../figures/stochastic_env_x_t.pdf}
\caption{Stochastic local stability. The figure shows the frequency of
phenotype \emph{A} after \(10^6\) generations in a very large population 
evolving in a stochastic environment (eq.\ 31). The
fitness of phenotypes \emph{A} and \emph{B} are \(1+s_t\) and \(1\),
where \(s_t\) is \emph{s} with probability \emph{p} and \emph{-s} with
probability \(1-p\). The white line marks combinations of \emph{p}
and \emph{s} for which \(E[\log{(1+\rho s_t)}]=0\); according
to Result 6, below this line fixation of phenotype \emph{B} (\(x^*=0\)) will be
stochastically locally stable. Parameters: initial frequency of \(A\) is \(x_0=0.1\); vertical transmission rate is \(\rho=0.1\).}\label{fig:stochastic_env_x_t}
\end{figure}

%Figure 5
\begin{figure}
\centering
\includegraphics{../figures/stoch_two_iid_rvs.png}
\caption{Effect of vertical transmission rate $\rho$ on phenotype polymorphism in a randomly changing environment. Dynamics of the frequency of phenotype \(A\) over time starting at \(x_0=10^{-5}\) when the fitness of phenotypes \(A\) and \(B\) are identically and independently distributed random variables.
As the vertical transmission rate \(\rho\) increases from 0.001 to 0.5 the frequency reaches a polymorphic distribution with \(E(x_t)\to0.5\) faster,
but the variance also increases.
The fitness of phenotypes \(A\) and \(B\), \(w_A\) and \(w_B\), are both exponential random variables with expected value 2.}\label{fig:stoch_two_iid_rvs}
\end{figure}

%Figure 6
\begin{figure}
\centering
\includegraphics{../figures/A1B1_modifier_invasions.pdf}
\caption{Consecutive fixation of modifiers that reduce the vertical
transmission rate in environmental regime \emph{A1B1}. The figure shows
results of numerical simulations of evolution with two modifier alleles
(eq.\ 35). When a modifier allele fixes
(frequency\textgreater{}99.9\%), a new modifier allele is introduced
with a vertical transmission rate one order of magnitude lower (vertical
dashed lines). \textbf{(A,D,G)} The frequency of phenotype \emph{A} in
the population over time. \textbf{(B,E,H)} The frequency of the invading
modifier allele over time. \textbf{(C,F,I)} The population mean
fitness over time; insets zoom in to show that the mean fitness 
decreases slightly with each invasion. Parameters: vertical transmission rate of
the initial resident modifier allele, \(\rho_0 =0.1\); fitness values:
$W=1$, $w=0.1$ (\textbf{A-C}), 0.5 (\textbf{D-F}), and 0.9
(\textbf{G-I}). The x-axis is on a log-scale, as each sequential invasion
takes an order of magnitude longer to
complete.}\label{fig:A1B1_modifier_invasions}
\end{figure}

%Figure 7
\begin{figure}
\centering
\includegraphics[height=\dimexpr \textheight - 8\baselineskip\relax]{../figures/stoch_modifier_invasions.png}
\caption{Consecutive fixation of modifiers that reduce the vertical
transmission rate $\rho$ under symmetric randomly changing selection. The figure shows
results of numerical simulations of evolution with two modifier alleles
(eq.\ 33). When a modifier allele fixes
(frequency\textgreater{}99.9\%), a new modifier allele is introduced
with a vertical transmission rate one order of magnitude lower (vertical
dashed lines). \textbf{(A)} The frequency of phenotype \emph{A} in the
population over time. \textbf{(B)} The frequency of the invading
modifier allele over time. Parameters: vertical transmission rate of the initial
resident modifier allele, \(\rho_0 =0.1\); fitness values: $W=1$
and $w=0.1$ with probability 0.5 and $w=0.1$ and $W=1$
also with probability 0.5. The x-axis is on a log-scale, as each sequential invasion takes
an order of magnitude longer to
complete.}\label{fig:stoch_modifier_invasions}
\end{figure}

%Figure 8
\begin{figure}
\centering
\includegraphics{../figures/A1B2_polymorphism.pdf}
\caption{Stable frequency of phenotype \emph{A} and population mean fitness in selection regime
$A1B2$ as a function of the vertical transmission rate \(\rho\) and the fitness ratio \(W/w\) between the favored and unfavored phenotypes.
In the grey area \(B\) reaches fixation and the frequency of \(A\) is 0.
\textbf{(A)} Stable frequency of phenotype $A$ at the end of each period of three generations.
\textbf{(B)} Geometric average of the stable mean fitness over the three-generation period: $(\bar{w}^* \cdot \bar{w}^{**} \cdot \bar{w}^{***})^{1/3}$.
In all cases, \emph{W=1}.}\label{fig:A1B2_polymorphism}
\end{figure}

%Figure 9
\begin{figure}
\centering
\includegraphics{../figures/rho_evol_stability.pdf}
\caption{Evolutionarily stable vertical transmission rate in $AkBk$ selection regime. 
The evolutionary stable rate $\rho^*$ was determined as in Carja et al.\ (2014) using simulations of eq.\ 35: for each $k$ from 1 to 50, ten initial rates were randomly chosen between 0 and 1 for the resident modifier $m$; after the population evolved with only $m$ for 1,000 generations towards an equilibrium, invader modifier alleles $M$ were consecutively introduced at frequency $10^{-4}$; their value for $\rho$ was a product of the resident rate and a random number from an exponential distribution with a mean of 1; if an invader increased in frequency after $5,000 \times 2k$ generations, the invader was considered successful and became the resident modifier for the next invasion, and allowed to evolve for 1,000 generations to reach a new equilibrium before the next invasion; if the invader was unsuccessful, a new invader was drawn. After at least 500 invasion attempts and 50 unsuccessful ones, the evolutionary process stopped. The markers show the average for ten such processes for each $k$ value. 95\% confidence intervals (calculated with bootstrap, 1,000 resamples) are too small to be seen. The dashed line shows the relation $\rho=1-1/(k-1)$ (Carja et al. 2014), which is close to the simulation results for \(w=0.5\).
In all cases, $W=1$.}\label{fig:rho_evol_stability}
\end{figure}

%Figure 10
\begin{figure}
\centering
\includegraphics{../figures/AkBk_stable_wbar_slice.pdf}
\caption{Vertical transmission rates that maximize population mean fitness in the \emph{AkBk} selection regime.
Populations evolved with different vertical transmission rates \(\rho\) and in different \emph{AkBk} selection regimes for \(1000 \times 2k\) generations, and the geometric average of the population mean fitness in the final \(2k\) generations was calculated.
For each \(k\), the rate \(\hat{\rho}\) that maximizes the geometric average is shown.
In all cases, $W=1$. See Fig. S3 for a broader view.} \label{AkBk_stable_wbar_slice}
\end{figure}
 
%Figure 11
\begin{figure}
\centering
\includegraphics{../figures/fixation_prob_time.pdf}
\caption{Fixation probability and time in a finite population.
\textbf{(A)} Fixation probability \(u(x)\) of phenotype \(A\) (eq.\ 70), and \textbf{(B)}
Expected time to fixation \(T(x)\) of phenotype \(A\) (eq.\ 72), conditioned on its
fixation, starting with a single copy in a population of size \(N\). The
figure compares two estimates: simulations (blue circles) and diffusion
equation approximation (green solid line). Parameters: Selection coefficient, \(s=w_A-w_B=0.1\);
Population size, \(N=10,000\).}\label{fixation_prob_time}
\end{figure}

%Figure 12
\begin{figure}
\centering
\includegraphics{../figures/lk_fix_prob.pdf}
\caption{Fixation in a finite population with different ratios of selection periods \(\frac{k}{l}\). Fixation probability of phenotype $A$ when starting with a single copy in a population of size $N$: $u(1/N) = (1-\exp(-2 \rho \frac{k-l}{k+l}(W-w))/(1-\exp(-2 N \rho \frac{k-l}{k+l}(W-w))$ (see eqs. 115-116).
\emph{k} and \emph{l} are the number of
generations in which phenotypes \emph{A} and \emph{B}, respectively, are favored by
selection. In all cases, $W = 1$, $w = 1-s$, Population size, \(N=10,000\).} \label{lk_fix_prob}
\end{figure}

\end{document}  