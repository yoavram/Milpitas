\magnification=1170
\hsize=6.5truein
\hoffset=.05truein
%\vsize=8.5truein
%\voffset=.25truein
\hfuzz=14pt
\parskip=5pt
\baselineskip=16pt

\def\harr#1#2{\smash{\mathop{\hbox to .2in{\rightarrowfill}}\limits^{\scriptstyle#1}_{\scriptstyle#2}}}

\def\var{\varepsilon}

\def\cl{\hbox{\tenbf L}}
\def\ci{\hbox{\tenbf I}}


\font\tit=cmbx10 at 14pt
\ \vskip1.1in
\centerline{\tit Vertical and Oblique Transmission under}\smallskip
\centerline{\tit Fluctuating Selection}
\vskip.3in


\centerline{Yoav Ram$^1$, Uri Liberman$^{2}$, and Marcus W. Feldman$^1$}


\vskip.5truein

%\centerline{$*$ Corresponding author}
\bigskip
\centerline{$^1$Department of Biology}
%\vskip-7pt
\centerline{Stanford University}
%\vskip-7pt
\centerline{Stanford, CA 94305-5020}
\bigskip

\centerline{$^2$School of Mathematical Sciences}
%\vskip-7pt
\centerline{Tel Aviv University}
%\vskip-7pt
\centerline{Tel Aviv, Israel 69978}
%\vskip-7pt
\centerline{uril@tauex.tau.ac.il}
\vfil
\centerline{July 27, 2017, v.2}

\break



\noindent{\bf THE MODEL}
\smallskip

Consider an infinite population whose members are characterized by their phenotype $\phi$, which can be $\phi=A$ or $\phi=B$, with associated frequencies $x$ and $1-x$, respectively.
We follow the evolution of $x$ over discrete non-overlapping generations.
In each generation individuals are subject to selection where the fitness values of $A$ and $B$ are $w_A$ and $w_B$, respectively, such that the frequency of phenotype $A$ after selection is
$$
x^S = {w_A\over \overline w}x,
\eqno(0)$$
where $\overline w$ is the {\sl mean fitness}, namely
 $$\overline w = w_A x+w_B(1-x).\eqno(3)$$
 
An offspring inherits its phenotype from its parent, via {\sl vertical transmission}, with probability $\rho$,
and from a random individual in the parental population, via {\sl oblique transmission}, with probability $1-\rho$.
Therefore, given that the parent phenotype is $\phi$ and assuming {\sl uni-parental inheritance} (Zefferman 2016), 
the conditional probability that the phenotype $\phi'$ of the offspring is $A$ is
$$P(\phi' =A|\phi) =\left\{\matrix{(1-\rho)x+\rho, &\hbox{\tenrm if}\ \phi=A\cr 
\noalign{\smallskip}
(1-\rho)x, & \hbox{\tenrm if}\ \phi =B\cr}\right.\eqno(1)$$
where $x=P(\phi=A)$ in the parent's generation before selection.
 
Therefore, the change in frequency of phenotype $A$ after one generation due to selection ($ {w_A\over \overline w}$) and vertical ($\rho$) and oblique ($1-\rho$) transmission is given by the recursion equation
$$
x' = \rho  {w_A\over \overline w} x + (1-\rho) x
,\eqno(2)$$
which can be rewritten as 
$$\eqalign{
x' = 
\bigl(1-x\bigr) {w_A\over \overline w} \bigl((1-\rho)x + \rho\bigr) + \bigl(1-x\bigr) {w_B\over \overline w} \bigl(1-\rho\bigr)x = \cr
x \cdot \Bigl(1 - \rho + \rho {w_A \over \overline w}\Bigr) = \cr
x \cdot \Bigl(1 + \rho \bigl(1-x\bigr) {w_A-w_B \over \overline w} \Bigr)
.}\eqno(4)$$
 
 In what follows we explore the evolution of the recursion (4), namely the equilibria and their stability properties, in the cases of {\sl constant environments} and {\sl changing environments}.
 \bigskip
 \bigskip

\noindent{\bf CONSTANT ENVIRONMENT}
\smallskip

When the environment is constant, the fitness parameters $w_A$ and $w_B$ do not change between generations, and we have the following result.

\proclaim Result 1. If $0<\rho\le 1$ and both $w_A$ and $w_B$ are positive with $w_A\ne w_B$, then fixation in the phenotype $A(B)$ is globally stable when $w_A>w_B$ $(w_A<w_B)$.\par

\noindent{\it Proof}.
If we rewrite (4) as $x'=x\cdot F(x)$, it can be seen that $F(1)=1$, and for $\rho>0$ and $0<x<1$
$$\eqalign{
&F(x)>1\quad\hbox{\tenrm when}\quad w_A>w_B,\cr
&F(x)<1\quad\hbox{\tenrm when}\quad w_A<w_B.}\eqno(5)$$
Hence, as $w_A>0$ and $w_B>0$, both fixations in $A$ or in $B$  ($x^*=1$ for fixation in $A$ and $x^*=0$ for fixation in $B$) are equilibrium points of (4). Moreover, if $x_t$ is the value of $x$ at the $t$-th generation $(t=0,1,2,\dots)$, from (4)  and (5) we have for any $0<x_0<1$ and all $t=0,1,2,\dots$,
$$\eqalign{
&x^{t+1}>x^t\quad\hbox{\tenrm when}\quad w_A>w_B,\cr
&x^{t+1}<x^t\quad\hbox{\tenrm when}\quad w_A<w_B.}\eqno(6)$$

\noindent Hence, as $x^*=1$ or $x^*=0$ are the only equilibrium points, then
$$\eqalign{
&\lim_{t\to\infty}x_t=1,\ \ \hbox{\tenrm for all}\ 0<x_0\le 1,\ \ \hbox{\tenrm when}\ w_A>w_B,\cr
&\lim_{t\to\infty}x_t=0,\ \ \hbox{\tenrm for all}\ 0\le x_0< 1,\ \ \hbox{\tenrm when}\ w_A<w_B.}\eqno(7)$$
Thus fixation of the favored phenotype is globally stable, and ultimately all individuals have the favored phenotype.
\bigskip

\noindent{\it Remarks}

\item{1.} If $\rho=0$, there is perfect oblique transmission, and the recursion (4) reduces to $x'=x$; the phenotype frequencies do not change over time.
\item{2.} If $w_A=w_B>0$, again $x'=x$, and no stable equilibrium exists.
\item{3.}With {\sl extreme selection}, where for example $w_A=1$ and $w_B=0$ (or $w_B=1$ and $w_A=0$), individuals of the disfavored phenotype cannot reproduce and the recursion (4) reduces to
$$x'=(1-\rho)x+\rho.\eqno(8)$$
Therefore
$$x_t=1-(1-\rho)^t(1-x_0).\eqno(9)$$
Hence, unless $\rho=0$ we have $x_t\,\harr{}{t\to\infty}\,1$ for all initial values $0\le x_0\le 1$ and ultimately the favored phenotype fixes.\par

Figure 1(a) shows the transformation (4) and Figure 1(b) represents its dynamics for different vertical transmission rates $\rho$ when $w_A=1$ and $w_B=0.9$. Figure 2 shows the time to fixation $\tau$ (i.e., time to reach $x_t=1-\var$, for $\var$ small, which is $\tau=\log{\var\over1-x_0}/\log(1-\rho)\simeq\log\var/\log(1-\rho)$ for $x_0\approx0$) with extreme selection $(w_A=1, w_B=0)$ for different values of $\rho$ and initial frequency $x_0$. We see that greater vertical transmission leads to significantly lower fixation time.
\bigskip
\bigskip

\noindent{\bf PERIODICALLY CHANGING ENVIRONMENT}
\smallskip

Next we consider the case where the environment is not constant but rather changes periodically, such that the favored phenotype changes after a fixed number of generations. Simple examples are $A1B1=ABABAB\dots$, in which the favored genotype switches every generation, or $A2B1=AABAABAAB\dots$, where every two generations in which selection favors $A$ are followed by a single generation in which selection favors $B$.
 
 In general, $AkBl$ denotes a selection regime in which the period is of $(k+l)$ generations with $k$ generations favoring phenotype $A$ followed by $l$ generations favoring $B$.
 
 Let $W$ be the fitness of the favored phenotype and $w$ be that of the other phenotype where $0<w<W$. Rewrite equation (4) as $x'=F_A(x)=xf_A(x)$ when $A$ is favored and $x'=F_B(x)=xf_B(x)$ when $B$ is favored. Then
  $$\eqalign{
  &f_A(x) = {1 + \rho (1-x) {W-w \over Wx+w(1-x)}},\cr\noalign{\smallskip}
  &f_B(x) = {1 + \rho (1-x) {w-W \over wx+W(1-x)}}.}\eqno(10)$$
  
  \noindent If $x_t$ denotes the frequency of phenotype $A$ at generation $t$ starting with $x_0$ initially,
  then as we are interested in the values of $x_t$ for $t=n(k+l)$ with $n=0,1,\dots$ at the end of complete periods then in fact
  $$x_{(n+1)(k+l)}=F(x_{n(k+l)}),\quad n=0,1,2,\dots,\eqno(11)$$
  where $F$ is the composed function
  $$F=\underbrace{F_B\circ F_B \circ\cdots\circ F_B}_{l\ \hbox{\sevenrm times}} \circ\underbrace{F_A\circ F_A \circ\cdots\circ F_A}_{k\ \hbox{\sevenrm times}}.\eqno(12)$$
  Clearly, since $F_A(0) =F_B(0) =0$ and $F_A(1) =F_B(1) =1$, both fixations in $A$ or in $B$ are equilibrium points. An interesting question is when these fixations are locally stable (. We concentrate on $x^*=0$, the fixation of the phenotype $B$. As $x'=F_A(x) =xf_A(x)$ for $k$ generations and $x'=F_B(x) =xf_B(x)$ for $l$ generations, the linear approximation of $F(x)$ ``near'' $x=0$ is
   $$F(x) \simeq\bigl[f_A(0)\bigr]^k \bigl[f_B(0)\bigr]^lx.\eqno(13)$$
   Hence the local stability of $x^*=0$ is determined by the product $\bigl[f_A(0)\bigr]^k \bigl[f_B(0)\bigr]^l$, and $x^*=0$  is locally stable if the product in (13) is less than 1 and unstable if it is larger than 1.
   
   From (10) we have
   $$\eqalign{
   &f_A(0) = 1+\rho{W-w\over w},\cr\noalign{\smallskip}
   &f_B(0) = 1+\rho{w-W\over W}.}\eqno(14)$$
   We start with the case $k=l$.
   
   \proclaim Result 2. If $k=l$ and $0<w<W$ with $0<\rho<1$, fixation of $B$ is unstable.\par
   
   \noindent {\it Proof.} The local stability of $x^*=0$, the fixation of $B$, is determined by the product
   $$\bigl[f_A(0)\bigr]^k \bigl[f_B(0)\bigr]^k=\left[\left(1+\rho{W-w\over w}\right)\left(1+\rho{w-W\over W}\right)\right]^k.\eqno(15)$$
   Observe that
   $$\left(1+\rho{W-w\over w}\right)\left(1+\rho{w-W\over W}\right) =1+\rho(1-\rho){\left(W-w\right)^2\over wW}>1.\eqno(17)$$
   Since $0<\rho<1$ and $0<w<W$,  fixation of $B$ is unstable.
  \bigskip

\noindent{\it Conclusions}

\item{1.}As $k=l$ and the above result holds also when $0<W<w$, there is total symmetry between the two fixations in $A$ and $B$, and fixation in $A$ is not stable. Thus neither phenotype can become extinct, and we have a {\sl protected polymorphism} (Prout 1968).

\item{2.}Observe that for general $k$, $l$, the condition for local stability of fixation in $A$ is
$$\bigl[f_A(0)\bigr]^l \bigl[f_B(0)\bigr]^k <1, \eqno(18)$$
and that of $B$ is
$$\bigl[f_A(0)\bigr]^k \bigl[f_B(0)\bigr]^l <1. \eqno(19)$$
Therefore, following Result 2, 
$$\bigl[f_A(0)\bigr]^{k+l}\bigl[f_B(0)\bigr]^{k+l}>1,\eqno(20)$$
and it is impossible that both fixations are stable. Further, since by (14)  $f_A(0)>1$ and $0<f_B(0)<1$ when $0<w<W$, by choosing $k$ and $l$ appropriately, fixation on $A$ or fixation on $B$ (but not both) can be stable. In addition, we can have both fixations unstable giving the following result.

\proclaim Result 3. With $0<\rho<1$ and $0<w<W$ in the case of $AkBl$ periodically changing environments, both fixations may be unstable, producing a protected polymorphism.\par

\noindent{\it Proof}. Let $a=1+\rho{W-w\over w}$ and $b=1+\rho{w-W\over W}$. Then since $0<\rho <1$ and $0<w<W$, we have $a>1$ and $0<b<1$. Following (13), fixation in $B$ is not stable if $a^kb^l>1$, and similarly  fixation in $A$ is unstable if $a^lb^k>1$. Therefore both fixations are unstable if
$$a^kb^l>1\quad\hbox{\tenrm and}\quad a^lb^k>1,\eqno(21)$$
or equivalently if
$$k\log a +l\log b>0\quad\hbox{\tenrm and}\quad l\log a+k\log b>0.\eqno(22)$$
Now (22) holds if and only if
$$k\,{\log(1/b)\over\log a} < l < k\,{\log a\over\log(1/b)}.\eqno(23)$$ 
Inequalities (23) are consistent if and only if $\log(1/b)<\log a$, i.e., $ab>1$.
Indeed, by (17),
$$ab=\left(1+\rho{W-w\over w}\right)\left(1+\rho{w-W\over W}\right)>1.\eqno(24)$$

\noindent{\it Remark}.
It should be pointed out that the local stability properties of the two fixations depend only on the fact that in a cycle of $(k+l)$ generations $A$ was favored $k$ times and $B$ was favored $l$ times, and not their order in the cycle.


When neither fixation in $A$ or $B$ is stable, we have a case of protected polymorphism, and we expect to have one or more polymorphic equilibria. For the simple case of $A1B1$ periodically changing environment we have


\proclaim Result 4. In the case $A1B1$ with $0<\rho<1$ and $0<w<W$ the two fixations are unstable and there exists a unique stable polymorphism.

\noindent{\it Proof}. Let $x$ be the initial frequency of $A$ and $x'$ its frequency  after one cycle of the $A1B1$ selection. Then $x' =F_B\bigl(F_A(x)\bigr)$ where by (10)
$$\eqalign{
&F_A(x) = x \Bigl(1 + \rho (1-x){W-w \over Wx+w(1-x)}\Bigr),\cr\noalign{\smallskip}
&F_B(y) = y \Bigl(1 + \rho (1-y){w-W \over wy+W(1-y)}\Bigr).}\eqno(25)$$
The equilibrium equation is $x=F_B\bigl(F_A(x)\bigr)$, which reduces to a fourth degree polynomial equation in $x$. Since the two fixations in $B$ and $A$ are equilibria corresponding to the two solutions $x=0$ and $x=1$, the other equilibria correspond to solutions of a quadratic equation $G(x) =Ax^2 +Bx +C=0$ with
$$A=1,\quad B={W+w\over (2-\rho)(W-w)}-1,\quad C={-w\over (2-\rho)(W-w)}.\eqno(26)$$
As $0<\rho<1$ and $0<w<W$ we have
$$G(0) = {-w\over (2-\rho)(W-w)}<0,\eqno(27)$$
and
$$G(1)=A +B+ C =
{W\over (2-\rho)(W-w)} >0,\eqno(29)$$
and $G(x)$ must have two real roots.
Also, since $A=1$, $G(x)$ is convex, and therefore there is one negative root, and one positive root $x^*$ which satisfies $0<x^*<1$.
The latter root determines a unique polymorphism.

Let $H(x) =F_B\bigl(F_A(x)\bigr)$. Then 
 $$H(0)=0,\quad H(x^*)=x^*,\quad H(1)=1.\eqno(30)$$
 Also 
 $$F'_A(x) = {x^2(1-\rho)\left(W-w\right)^2 +2xw(1-\rho)(W-w) +w\left[\rho W +(1-\rho)w\right]\over \left[x(W-w) +w\right]^2}\eqno(31)$$
 and
  $$F'_B(x) = {x^2(1-\rho)\left(w-W\right)^2 +2xW(1-\rho)(w-W) +W\left[\rho w +(1-\rho)W\right]\over \left[x(w-W) +W\right]^2}.\eqno(32)$$
  As $0<\rho<1$ and $0<w<W$, we have $F'_A(x)>0$ for  $0\le x\le 1$. Observe that the numerator of $F'_B(x)$ is linear in $\rho$; its value when $\rho=1$ is $wW>0$, and when $\rho=0$ it is
  $$x^2\left(w-W\right)^2 +2xW(w-W) +W^2 =\left[x(w-W) +W\right]^2 >0.\eqno(33)$$
  Hence $F'_{B(x)}>0$ for all $0\le x\le 1$, and $H'(x) =F'_B\bigl(F_A(x)\bigr)F'_A(x)$ is positive when $0\le x\le 1$. Thus $H(x)$ is a monotone increasing function of $x$ for $0\le x\le 1$.
  
  As the two fixations are unstable, combining all the above properties, $H(x)$ has the following graph.
  
  \bigskip\centerline{[GRAPH HERE]}\bigskip
  
  \noindent Therefore $x^*$, the unique polymorphism, is globally stable. That is, starting from any initial value $0<x_0 <1$ we have  $x_t\,\harr{}{t\to\infty}\,x^*$.
  
 \bigskip
 \bigskip
 
 \noindent{\bf RANDOMLY CHANGING ENVIRONMENT}
 \smallskip
 
 Now consider the  case where the environment changes according to a stochastic process. Without loss of generality, assume that the fitness parameters at generation $t$ ($t=0,1,2,\dots$) are $1+s_t$ for phenotype $A$ and $1$ for phenotype $B$, such that the random variables $s_t$ for $t=0,1,2,\dots$ are independent and identically distributed. Also we assume that there are positive constants $C$ and $D$ such that $P(C<s_t <D)=1$.
 
 Corresponding to (4), the recursion equation is
 $$x_{t+1}=x_t {1+\rho s_t +x_t(1-\rho)s_t\over 1+x_ts_t}\quad t=0,1,2,\dots\; .\eqno(34)$$
 
 As $\{x_t\}$ for $t=0,1,2,\dots$ is a sequence of random variables, the notion of stability of the two fixation states needs clarification. Following Karlin and Liberman (1975) we make the following definition.
 
 \noindent{\it Definition}: ``stochastic local stability''.
 A constant equilibrium state $x^*$ is said to be {\sl stochastically locally stable} if for any $\var>0$ there exists a $\delta>0$ such that $|x_0-x^*|<\delta$ implies
 $$P\left(\lim_{t\to\infty}x_t =x^*\right)\ge 1-\var.\eqno(35)$$
 Thus stochastic local stability holds for $x^*$ provided for any initial $x_0$ sufficiently near $x^*$ the process $x_t$ converges to $x^*$ with high probability.
 
 In our case there are two constant equilibria $x^*=0$ and $x^*=1$ corresponding to fixation in $B$ and $A$, respectively. Following Karlin and Liberman (1975) we can characterize the stochastic local stability of these fixations as follows.
 
 \proclaim Result 5. Suppose $E\left[\log (1+\rho s_t)\right]>0$. Then $x^*=0$, the fixation in $B$, is not stochastically locally stable. In fact $P\left(\lim_{t\to\infty}x_t=0\right)=0$.
 
 \noindent{\it Proof}. Rewrite recursion (34) as
 $${x_t+1\over x_t} =(1+\rho s_t)\left[1-x_t{\rho s_t(1+s_t)\over (1+\rho s_t)(1+x_ts_t)}\right].\eqno(36)$$
 Then
 $$\log x_{t+1} -\log x_t =\log(1+\rho s_t) +\log\left[1-x_t{\rho s_t(1+s_t)\over (1+\rho s_t)(1+x_ts_t)}\right].\eqno(37)$$
 Summation yields
 $${1\over t}\left[\log x_t-\log x_0\right] ={1\over t}\sum_{n=0}^{t-1}\log(1+\rho s_n) +{1\over t}\sum_{n=0}^{t-1}\log\left[1-x_n{\rho s_n(1+s_n)\over (1+\rho s_n)(1+x_ns_n)}\right].\eqno(38)$$
Let $\mu=E\left[\log(1+\rho s_t)\right]$. As $\{s_t\}_{t\ge 0}$ are independent and identically distributed random variables, the {\sl strong law of large numbers} applies and
$$\lim_{t\to\infty}{1\over t}\sum_{n=0}^{t-1}\log(1+\rho s_n)=\mu\eqno(39)$$
almost surely.

Let $\zeta$ be such that ${1\over t}\sum_{n=0}^{t-1}\log[1+\rho s_n(\zeta)]=\mu$ and assume that $\lim_{t\to\infty}x_t(\zeta)=0$. As the random variables $\{s_t\}_{t\ge 0}$ are uniformally bounded,
$$x_t(\zeta){\rho s_t(\zeta)[1+s_t(\zeta)]\over [1+\rho s_t(\zeta)][1+x_t(\zeta)s_t(\zeta)]}\;\;\harr{}{t\to\infty}\;\;0\eqno(40)$$
and
$$\lim_{t\to\infty}{1\over t}\sum_{n=0}^{t-1}\log\left[1-x_n(\zeta){\rho x_n(\zeta)[1+s_n(\zeta)]\over [1+\rho s_n(\zeta)][1+x_t(\zeta)s_n(\zeta)]}\right]=0.\eqno(41)$$
Thus (38) implies that
$$\lim_{t\to\infty}{1\over t}\left[\log x_t(\zeta) -\log x_0(\zeta)\right]=\mu.\eqno(42)$$
If $\mu =E\left[\log(1+s_t)\right]>0$, then from (42) we deduce that $\lim_{t\to\infty}x_t(\zeta)=\infty$, a contradiction.
Therefore when $\mu>0$, $P\left(\lim_{t\to\infty}x_t=0\right)=0$, and fixation of $B$ ($x^*=0$) is stochastically locally unstable.

 Thus by Result 5, for $x^*=0$ to be stochastically locally stable it is necessary that $E[\log(1+\rho s_t)]\le 0$. In fact, the strict inequality is sufficient.
 
 \proclaim Result 6. Suppose $E[\log(1+\rho s_t)]<0$. Then $x^*=0$ is stochastically locally stable.
 
 \noindent{\it Proof}. Let $\mu=E[\log(1+\rho s_t)]$. Then as $\{s_t\}_{t\ge 0}$ are independent and identically distributed random variables, the strong law of large number applies and almost surely
 $$\lim_{t\to\infty}{1\over t}\sum_{n=0}^{t-1}\log(1+\rho s_n)=\mu<0.\eqno(43)$$
 Appealing to the Egoroff Theorem, for any $\var>0$ there exists $T$ such that 
 $$P\left({1\over t}\sum_{n=0}^{t-1}\log(1+\rho s_n)<{\mu\over 2}\ \hbox{\tenrm for all}\ t\ge T\right)\ge 1-\var.\eqno(44)$$
 As $0\le \rho\le 1$ and the $\{s_t\}_{t\ge 0}$ are uniformly bounded, we can find a $\delta'>0$ such that
 $$x_t<\delta'\Longrightarrow\left|\log\left[1-x_t{\rho s_t(1+s_t)\over (1+\rho s_t)(1+x_ts_t)}\right]\right|<-{\mu\over 4}.\eqno(45)$$
 Also, as $0\le x_t\le 1$ for all $t$,
 $$x_{t+1}=x_t{1+\rho s_t +x_t(1-\rho)s_t\over 1+x_ts_t}< Kx_t,\eqno(46)$$
 where $K$ is independent of $t$. It follows that there exists a $\delta$ with $0<\delta<\delta'$ such that
 $$x_o<\delta\Longrightarrow x_t<\delta'\ \hbox{\tenrm for all}\ t=0,1,2,\dots,T-1.\eqno(47)$$
 Let $\xi$ be a realization of the evolutionary process such that
 $${1\over t}\sum_{n=0}^{t-1}\log[1+\rho s_n(\xi)]<{\mu\over 2}\ \hbox{\tenrm for all}\  t\ge T\eqno(48)$$
 and assume $x_0<\delta$. Then
 $$\eqalign{
 {1\over T}&[\log x_T(\xi) -\log x_0(\xi)] \cr
 &={1\over T}\sum_{n=0}^{T-1}\log[1+\rho s_n(\xi)]+{1\over T}\sum_{n=0}^{T-1}\log[1-x_n(\xi){\rho s_n(\xi)[1+s_n(\xi)]\over [1+\rho s_n(\xi)][1+x_n(\xi)s_n(\xi)]}\cr\noalign{\smallskip}
 &<{\mu\over 2}- {\mu\over 4} ={\mu\over 4}<0, }\eqno(49)$$
 and therefore $x_T(\xi) <x_0(\xi) <\delta'$. Invoking induction we get that for $t\ge T$
 $${1\over t}\log{x_t(\xi)\over x_0}\le {\mu\over 4},\eqno(50)$$
 or for all $t\ge T$
 $$x_t(\xi)\le x_0\exp\left({\mu\over 4}t\right).\eqno(51)$$
 
 As $\mu<0$, this implies that $x_t(\xi)\,\harr{}{t\to\infty}\,0$. Therefore we have shown that for given $\var>0$ there is a $\delta>0$ such that if $0<x_0<\delta$, then $P\left(\lim_{t\to\infty}x_t =0\right)\ge 1-\var$, and therefore $x^*=0$, the fixation in $B$ is stochastically locally stable.
\smallskip
 
 \noindent{\it Remark}. 
 Using the general notation for the fitness parameters $w_A$ and $w_B$, the stochastic local stability of the $B$ fixation is determined by the sign of $E\left[\log\left(1+\rho{w_A-w_B\over w_B}\right)\right]$ and that of the $A$ fixation by the sign of $E\left[\log\left(1+\rho{w_B-w_A\over w_A}\right)\right]$. When one of these signs is negative, there is stochastic local stability, and when it is positive, with probability one convergence to that fixation does not occur. Following (17),
 $$\left(1+\rho{w_A-w_B\over w_B}\right)\left(1+\rho{w_B-w_A\over w_A}\right) >1.\eqno(52)$$
 Therefore, as in the case of periodically changing environments $AkBl$, it is impossible that the two fixations are both stochastically locally stable.
 \bigskip
 \bigskip
 
 \noindent{\bf EVOLUTIONARY STABILITY OF OBLIQUE TRANSMISSION}
 \smallskip
 
 An interesting question concerns the evolution of oblique transmission itself. For example, is there an evolutionarily stable rate of oblique transmission? To answer this question we use a modifier model.
 
 Suppose that the vertical transmission rate is determined by a genetic locus with two possible alleles $m$ and $M$. Let the vertical transmission rates determined by $m$ and $M$ be $\rho$ and $P$, respectively. Thus there are four pheno-genotypes $mA$, $mB$, $MA$, and $MB$ whose frequencies at a given generation are denoted by $x_1$, $x_2$, $x_3$, and $x_4$, respectively. As the fitness values are determined by the two phenotypes $A$ and $B$, and the modifier locus is selectively neutral, we have the following table.
 $$\matrix{
 \hbox{\tenrm pheno-genotype}&mA&mB&MA&MB\cr\noalign{\smallskip}
 \hbox{\tenrm frequency}&x_1&x_2&x_3&x_4\cr\noalign{\smallskip}
 \hbox{\tenrm fitness}&w_A&w_B&w_A&w_B\cr\noalign{\smallskip}
 \hbox{\tenrm vertical transmission rate}&\rho&\rho&P&P\cr}\eqno(53)$$
 Following the rationale leading to equation (2), the pheno-genotype frequencies after one generations are described by
 $$\eqalign{
 \overline w x_1' = w_A x_1\bigr[(1-\rho)(x_1+x_3)+\rho \bigl] + w_B x_2(1-\rho)(x_1+x_3)\cr
 \overline w x_2' = w_A x_1(1-\rho)(x_2+x_4) + w_B x_2 \bigl[(1-\rho)(x_2+x_4) + \rho \bigr]\cr
 \overline w x_3' = w_A x_2\bigr[(1-P)(x_1+x_3)+P \bigl] + w_B x_4(1-\rho)(x_1+x_3)\cr
 \overline w x_2' = w_A x_3(1-P)(x_2+x_4) + w_B x_4 \bigl[(1-P)(x_2+x_4) + P \bigr]
}\eqno(54)$$
with the mean fitness $\overline w$ given by
$$\overline w=w_A (x_1+x_3) + w_B (x_2+x_4).\eqno(55)$$

Starting with a stable equilibrium where only the modifier allele $m$ is present, we check its {\sl external stability} to invasion by the modifier allele $M$.

Because we assume the initial existence of a stable polymorphic equilibrium that depends on $\rho$ (the rate determined by $m$), we cannot assume a {\sl constant environment}, which always leads to fixation of the favored type, independent of $\rho$. We therefore assume changing environments, and, in particular, the simple case of the $A1B1$ cycling environment, where a unique stable polymorphism exists and depends on $\rho$. Specifically, following (54-55) with $w_A =W$, $w_B =w$ in the first generation and $w_A=w$, $w_B=W$ in the second generation, after two generations we have
$$\underline x''=T_2(T_1\underline x),\eqno(56)$$
where $\underline x'=T_1\underline x$ is given by (54-55) with $w_A=W$, $w_B=w$, and $\underline x''=T_2\underline x'$ is given by (54-55) with $w_A=w$, $w_B=W$. Here $\underline x$, $\underline x'$, $\underline x''$ are the frequency vectors.

From the analysis of the $A1B1$ case, we know that when only the $m$ allele is present with associated rate $\rho$, with $0<\rho<1$ and $0<w<W$, a unique stable equilibrium $\underline x^* =(x_1^*,x_2^*,0,0)$ exists. $x_1^*$ is the only positive root of the quadratic equation $G(x) =Ax^2 +Bx +C=0$ with $A,B,C$ specified in (26). Solving $G(x)=0$ gives
$$x_1^* ={1\over 2} -{W+w-\sqrt{\left(1-\rho\right)^2\left(W-w\right)^2 +4Ww}\over 2\cdot(2-\rho)(W-w)},\eqno(57)$$
 and it can be seen that
 $${\sqrt{Ww}-w\over W-w}<x_1^* <{1\over 2}.\eqno(58)$$
 
 The external stability of $\underline x^*$ to the introduction of the modifier allele $M$ with rate $P$ is determined by the linear approximation matrix $\cl=\cl_2\cdot\cl_1$ derived from (54-55) and given by
 $$\overline w^*\cl_1 =\left[\matrix{W\bigl[(1-P)x_1^* +P\bigr] & w(1-P)x_1^*\cr\noalign{\medskip}
 W(1-P)x_2^* & w\bigl[(1-P)x_2^* +P\bigr]\cr}\right]\eqno(59)$$
 and
 $$\overline w^{**}\cl_2 =\left[\matrix{w\bigl[(1-P)x_1^{**} +P\bigr] & W(1-P)x_1^{**}\cr\noalign{\medskip}
 w(1-P)x_2^{**} & W\bigl[(1-P)x_2^{**} +P\bigr]\cr}\right],\eqno(60)$$
where $\underline x^{**} =T_1\underline x^*$ and
$$\overline w^* =Wx_1^* +wx_2^*,\qquad \overline w^{**} =wx_1^{**} +Wx_2^{**}.\eqno(61)$$
Due to the symmetry between the two phenotypes $A$ and $B$ in the $A1B1$ case, we have $x_1^{**} =1-x_1^*=x_2^*$ and $x_2^{**} =1-x_2^*=x_1^*$ so that $\overline w^{**} =\overline w^*$ and in fact $\cl_2$ can be written in terms of $x_1^*$ and $x_2^*$:
$$\overline w^*\cl_2 = \left[\matrix{w\bigl[(1-P)x_2^* +P\bigr] & W(1-P)x_2^*\cr\noalign{\medskip}
 w(1-P)x_1^* & W\bigl[(1-P)x_1^* +P\bigr]\cr}\right].\eqno(62)$$
  \smallskip
  
   \noindent{\it Remark}. 
 Observe that as $\underline x^* =T_2(T_1\underline x^*)$ with $x_3^*=x_4^*=0$, from (59) and (60) with $P=\rho$ we have
 $$\left[\matrix{x_1^*\cr\noalign{\medskip} x_2^*\cr}\right]=\cl\left[\matrix{x_1^*\cr\noalign{\medskip} x_2^*\cr}\right].\eqno(63)$$
 Hence when $P=\rho$, one of the eigenvalues of $\cl$ is 1.
 \smallskip
 
 In general, using (59) and (62) we get
 $$\eqalign{
 &\left(\overline w^*\right)^2\cl =\cr\noalign{\smallskip}
 &\left[\matrix{Ww\left[(1-P)^2x_1^*x_2^* +P\right] +\bigl[w(1-P)x_1^*\bigr]^2 & W(1-P)\bigl[P+x_1^*(1-P)\bigr]\bigl[Wx_2^* +wx_1^*\bigr]\cr
 \noalign{\medskip}
 w(1-P)\bigl[1-x_1^*(1-P)\bigr]\bigl[Wx_2^* +wx_1^*\bigr] & Ww\left[(1-P)^2x_1^*x_2^* +P\right] +\bigl[W(1-P)x_2^*\bigr]^2
 }\right].}\eqno(64)$$
 The external stability of $\underline x^*$ is determined by the eigenvalues of $\cl$, namely the roots of its characteristic polynomial $R(\lambda) =\det(\cl-\lambda\ci)$, with $\ci$ the $2\times 2$ identity matrix. From (64), $R(\lambda)=a_2\lambda^2 +a_1\lambda +a_0$, where
 $$a_0={P^2W^2w^2\over\left(\overline w^*\right)^4},\qquad a_1=-{2PWw +\left(1-P\right)^2\left[Wx_2^* +wx_1^*\right]^2\over \left(\overline w^*\right)^2},\qquad a_2=1.\eqno(65)$$
 
 We can prove the following result.
 
 \proclaim Result 7. $\cl$ has two positive eigenvalues and\hfil\break
  {\hglue.82truein}(i) when $P>\rho$ the two eigenvalues are less than 1,\hfil\break
  {\hglue.79truein}(ii) when $P<\rho$ the largest eigenvalue is larger than 1.\par
  
  \noindent{\it Proof}.  As $\cl$ is a positive matrix, by the Perron-Frobenius theory, $\cl$ has a positive eigenvalue.
  Also, the product of the two eigenvalues of $\cl$  is $a_0>0$.
  Thus $\cl$ has two positive eigenvalues.
  In addition, $R(\lambda)$ is convex because $a_2=1$.
  
  Let $R(1)=R(1;P)$ be a polynomial in $P$, then from (65)
  $$R(1;P) = -{\left(\overline w^*\widetilde w^*\right)^2 - \left(Ww\right)^2 \over\left(\overline w^*\right)^4}P^2 +
  2P{\left(\widetilde w^*\right)^2 -Ww\over \left(\overline w^*\right)^2} -
  {\left(\widetilde w^*\right)^2 - \left(\overline w^*\right)^2 \over\left(\overline w^*\right)^2},\eqno(66)$$
where $\widetilde w^* =Wx_2^* +wx_1^*$.  
By (58), $\bigl(\sqrt{Ww}-w\bigr)/(W-w) <x_1^* <{1\over 2}$, from which it can be seen that
$$\widetilde w^* >\overline w^* >\sqrt{Ww}.\eqno(67)$$

 When $P=\rho$, one of the eigenvalues of $\cl$ is 1; hence $R(1;\rho)=0$.
 Another root of $R(1;P)$ is $\bigl[(\overline w^*)^2 +\overline w^*\widetilde w^*\bigr]/\bigl[Ww +\overline w^*\widetilde w^*\bigr]$, which by (67) is larger than 1.
 As $R(1;0)= - \bigl[(\widetilde w^*)^2 - (\overline w^*)^2\bigr]/(\overline w^*)^2 <0$ by (67), and we deduce that when $0<P<\rho$, $R(1;P)<0$, whereas when $\rho<P<1$, $R(1;P)>0$.
 
 Hence, when $P<\rho$, we have $R(1)<0$ and $R(0)=a_0>0$, and we conclude that $R(\lambda)$ has a positive root larger than 1 and the therefore largest eigenvalue of $\cl$ is larger than 1.
 
 When $P>\rho$, we have $R(1)>0$ and $R(0)>0$. Also, because $R(\lambda)$ is convex and has two positive roots, these roots are either both less than 1 or both larger than 1.
 But the product of the two roots is $a_0=P^2{(Ww)^2 \over (\overline w^*)^4}<1$ by (67), and we conclude that when $P>\rho$, both roots are less than 1 and therefore the largest eigenvalue of $\cl$ is less than 1.
 \bigskip
 \bigskip
 
 \proclaim Conclusion.
 \smallskip
 
 A modifier allele $m$ producing vertical transmission rate $\rho$ is stable to the introduction of a modifier allele $M$ with associated rate $P$ if $P>\rho$, and it is unstable if $P<\rho$. Thus with the $A1B1$ environment cycle, evolution tends to reduce vertical transmission. There is therefore a {\sl Reduction Principle for Vertical Transmission} (Altenberg et al. 2017).
  
 \end\bye


