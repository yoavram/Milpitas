\documentclass[]{article}
\usepackage{lmodern}
\usepackage{amssymb,amsmath}
\usepackage{ifxetex,ifluatex}
\usepackage{fixltx2e} % provides \textsubscript
\ifnum 0\ifxetex 1\fi\ifluatex 1\fi=0 % if pdftex
  \usepackage[T1]{fontenc}
  \usepackage[utf8]{inputenc}
\else % if luatex or xelatex
  \ifxetex
    \usepackage{mathspec}
  \else
    \usepackage{fontspec}
  \fi
  \defaultfontfeatures{Ligatures=TeX,Scale=MatchLowercase}
\fi
% use upquote if available, for straight quotes in verbatim environments
\IfFileExists{upquote.sty}{\usepackage{upquote}}{}
% use microtype if available
\IfFileExists{microtype.sty}{%
\usepackage[]{microtype}
\UseMicrotypeSet[protrusion]{basicmath} % disable protrusion for tt fonts
}{}
\PassOptionsToPackage{hyphens}{url} % url is loaded by hyperref
\usepackage[unicode=true]{hyperref}
\urlstyle{same}  % don't use monospace font for urls
\usepackage{longtable,booktabs}
% Fix footnotes in tables (requires footnote package)
\IfFileExists{footnote.sty}{\usepackage{footnote}\makesavenoteenv{long table}}{}
\usepackage{graphicx,grffile}
\makeatletter
\def\maxwidth{\ifdim\Gin@nat@width>\linewidth\linewidth\else\Gin@nat@width\fi}
\def\maxheight{\ifdim\Gin@nat@height>\textheight\textheight\else\Gin@nat@height\fi}
\makeatother
% Scale images if necessary, so that they will not overflow the page
% margins by default, and it is still possible to overwrite the defaults
% using explicit options in \includegraphics[width, height, ...]{}
\setkeys{Gin}{width=\maxwidth,height=\maxheight,keepaspectratio}
\IfFileExists{parskip.sty}{%
\usepackage{parskip}
}{% else
\setlength{\parindent}{0pt}
\setlength{\parskip}{6pt plus 2pt minus 1pt}
}
\setlength{\emergencystretch}{3em}  % prevent overfull lines
\providecommand{\tightlist}{%
  \setlength{\itemsep}{0pt}\setlength{\parskip}{0pt}}
\setcounter{secnumdepth}{0}
% Redefines (sub)paragraphs to behave more like sections
\ifx\paragraph\undefined\else
\let\oldparagraph\paragraph
\renewcommand{\paragraph}[1]{\oldparagraph{#1}\mbox{}}
\fi
\ifx\subparagraph\undefined\else
\let\oldsubparagraph\subparagraph
\renewcommand{\subparagraph}[1]{\oldsubparagraph{#1}\mbox{}}
\fi

\newcommand{\beginsupplement}{%
	\setcounter{table}{0}
	\renewcommand{\thetable}{S\arabic{table}}%
	\setcounter{figure}{0}
	\renewcommand{\thefigure}{S\arabic{figure}}%
}


% set default figure placement to htbp
\makeatletter
\def\fps@figure{htbp}
\@fpsep\textheight
\makeatother

\usepackage{lineno}
\linenumbers

\begin{document}
\pagenumbering{gobble}
\beginsupplement

% 1
\begin{figure}
\centering
\includegraphics{../../figures/lk_phase_plane.pdf}
\caption{Ratios of selection periods \(\frac{k}{l}\) that lead to
fixation of phenotype \emph{A} (red) or polymorphism of phenotypes
\emph{A} and \emph{B} (blue). \emph{k} and \emph{l} are the number of
generations in which phenotypes \emph{A} and \emph{B}, respectively, are favored by selection.
In all cases, \emph{W} = 1, \emph{w} = \(1-s\).}\label{fig:lk_phase_plane}
\end{figure}

% 2
\begin{figure}
\centering
\includegraphics{../../figures/env_A1B1.pdf}
\caption{Frequency of phenotype \emph{A} after every two generations in
selection regime \(A1B1\). The orange line is the finite population model
(eqs. 52-53; average of 100 simulations). The blue line is the infinite population model
(eq.\ 21), and the green line is the  solution of \(Q(x)=0\) (eq.\ 22). In all cases, \(W=1\); for the finite population model (orange lines), population size is \(N=10,000\) and initial frequency of \(A\) is \(x_0=0.5\).}\label{fig:env_A1B1}
\end{figure}

% 3
\begin{figure}
\centering
\includegraphics{../../figures/A1B1_equilibrium.pdf}
\caption{Properties of stability in \emph{A1B1} selection regime.
\textbf{(A)} Stable frequency of phenotype $A$ and \textbf{(B)} stable mean fitness as functions of the vertical transmission rate \(\rho\) and the fitness of the disfavored phenotype \(w\).
Black contour lines join \(\rho\) and \(w\) combinations that result in the same stable value.
In all cases, fitness of the favored phenotype is \emph{W=1}.
}\label{fig:A1B1_equilibrium}
\end{figure}

% 4
\begin{figure}
\centering
\includegraphics{../../figures/AkBk_x0.pdf}
\caption{Convergence of the frequency of phenotype \emph{A} to a stable polymorphism in selection regime \emph{AkBk}.
Comparison of dynamics starting with different initial frequencies of phenotype \(A\) (0.01--0.99), and different $k$, $\rho$ and $w$ values.
The lines show the $x$ frequency of phenotype \emph{A} at the end of each period, after every $2k$ generations.
In all cases, $W=1$.}\label{fig:AkBk_x0}
\end{figure}

% 5
\begin{figure}
\centering
\includegraphics{../../figures/env_A1B2.pdf}
\caption{Frequency of phenotype \emph{A} after every three generations in
selection regime $A1B2$. Comparison of dynamics starting with
different initial frequency of phenotype \(A\) (0.01--0.99).
See also Figures 1 and  7. In all cases, $W=1$.}\label{fig:env_A1B2}
\end{figure}

% 6
\begin{figure}
\centering
\includegraphics{../../figures/env_A3B10.pdf}
\caption{Frequency of phenotype \emph{A} after every thirteen generations in
selection regime $A3B10$. Comparison of dynamics starting with
different initial frequency of phenotype \(A\) (0.01--0.99).
In all cases, $W=1$.}\label{fig:env_A3B10}
\end{figure}

% 7
\begin{figure}
\centering
\includegraphics{../../figures/AkBk_stable_wbar.pdf}
\caption{Stable population mean fitness in selection regime \emph{AkBk} as a function of the vertical transmission rate \(\rho\) and the number \(k\) of generations in which phenotypes \emph{A} and \emph{B} are favored by selection, for different selection intensities: \textbf{(A)} \(w=0.1\), \textbf{(B)} \(w=0.5\), and \textbf{(C)} \(w=0.9\).
Colors represent the geometric average of the stable population mean fitness over \(2k\) generations, calculated by iterating eq. 10 until phenotype frequencies stabilized, and for at least \(1,000 \) generations.
Blue markers show the maximum average mean fitness for each period \(k\).
For example, with \(w=0.1\), \(\hat{\rho}=0\) maximizes the average fitness for \(k \le 11\), then \(\hat{\rho}\) increases to \(\hat{\rho} \approx 0.24\), and then continues to decrease as \(k\) increases, down to \(\hat{\rho} \approx 0.15\) for \(k=50\) (see also Fig. 8A).
Contour lines represent \(\rho\) and \(k\) combinations that produce the same average mean fitness. 
In all cases, $W=1$.}\label{fig:AkBk_stable_wbar}
\end{figure}

% 8
\begin{figure}
\centering
\includegraphics{../../figures/AkBk_w0.5_geomwbar.pdf}
\caption{The geometric average of the stable population mean fitness over the $2k$ generation period peaks at $\rho=0$ for $k \le 30$ (red, blue and green lines) and at $\rho \approx 0.23$ for $k=31$ and $32$ (purple and orange lines). 
See also Figs. 8A, S5B. The inset zooms out to show that the geometric mean fitness is strictly and significantly decreasing for $\rho>0.3$ (reaching $\approx 0.7$ for $\rho=1$).
In all cases, $W=1$, $w=0.5$.}\label{fig:AkBk_w0.5_geomwbar}
\end{figure}

% 9
\begin{figure}
\centering
\includegraphics{../../figures/fixation_prob_time.pdf}
\caption{Fixation probability and time in a finite population.
\textbf{(A)} Fixation probability \(u(x)\) of phenotype \(A\) (eq.\ 55), and \textbf{(B)}
Expected time to fixation \(T(x)\) of phenotype \(A\) (eq.\ 56) conditioned on its
fixation, starting with a single copy in a population of size \(N\). The
figure compares two estimates: Wright-Fisher simulations (blue circles) and diffusion
equation approximation (green solid line). Parameters: selection coefficient, \(s=w_A-w_B=0.1\),
population size, \(N=10,000\).}\label{fixation_prob_time}
\end{figure}

% 10
\begin{figure}
\centering
\includegraphics{../../figures/AkBl_stable_modifier_w_0.1.pdf}
 \caption{Evolutionarily stable vertical transmission rate in $AkBl$ selection regime.
 The figure shows $\frac{\partial \lambda_1}{\partial P}$ the sensitivity of the leading eigenvalue of the external stability matrix $\mathbf{L}$ to changes in $P$ the vertical transmission rate of the invader allele as a function of $\rho$ the vertical transmission rate of the resident allele (see \emph{Section 2} in main text and \emph{Supplemental Material SP6} for details).
 The shaded area marks $\rho$ values for which phenotype $B$ fixes and there is no polymorphism (Eq.~(20) in main text).
 Without polymorphism, selection does not affect the transmission rate, so any rate in the shaded area is neutrally stable.
 In panels A, B, D, G, J, and M, $\frac{\partial \lambda_1}{\partial P} < 0$ at the vicinity of $\rho=0$ and therefore the stable rate is $\rho^*=0$.
 In panels B, C, E, F, H, I, K, and L, the stable rate $\rho^*$ can be identified as the $\rho$ value at which $\frac{\partial \lambda_1}{\partial P}$ changes from positive to negative.
 In panels N and O, $\frac{\partial \lambda_1}{\partial P} > 0$ for all $\rho$ values that protect polymorphism, so there are only neutrally stable rates (in the shaded area).
   Here, $W=1$ and $w=0.1$.}\label{fig:AkBl_stable_modifier_w_0.1}
\end{figure}

% 11
\begin{figure}
\centering
\includegraphics{../../figures/lk_fix_prob.pdf}
\caption{Fixation in a finite population with different ratios of selection periods \(\frac{k}{l}\). Fixation probability of phenotype $A$ when starting with a single copy in a population of size $N$: $u(1/N) = (1-\exp(-2 \rho \frac{k-l}{k+l}(W-w))/(1-\exp(-2 N \rho \frac{k-l}{k+l}(W-w))$ (see eqs. 59--60).
\emph{k} and \emph{l} are the number of
generations in which phenotypes \emph{A} and \emph{B}, respectively, are favored by
selection. In all cases, fitness of the favored phenotype, $W = 1$; fitness of the unfavored phenotype is $w=1-s$ and the population size is \(N=10,000\).} \label{lk_fix_prob}
\end{figure}

% 12
\begin{figure}
\centering
\includegraphics{../../figures/A1B1_modifier_invasions.pdf}
\caption{Consecutive fixation of modifiers that reduce the vertical
transmission rate in selection regime \emph{A1B1}. The figure shows
results of numerical simulations of evolution with two modifier alleles
(Eq. (32) in main text).
When a modifier allele fixes (frequency\textgreater{}99.9\%), a new modifier allele is introduced with a vertical transmission rate one order of magnitude lower (vertical
dashed lines). \textbf{(A,D,G)} The frequency of phenotype \emph{A} in
the population over time. \textbf{(B,E,H)} The frequency of the invading
modifier allele over time. \textbf{(C,F,I)} The population geometric mean
fitness over time; insets zoom in to show that the mean fitness 
decreases slightly with each invasion. 
Invading alleles are introduced at frequency 0.01\%; whenever their frequency drops below 0.01\% they are re-introduced.
Parameters: vertical transmission rate of
the initial resident modifier allele, \(\rho_0 =0.1\); fitness values:
$W=1$; $w=0.1$ (\textbf{A-C}), 0.5 (\textbf{D-F}), and 0.9
(\textbf{G-I}). The x-axis is on a log-scale, as each sequential invasion
takes an order of magnitude longer to complete.
Panels D-F are the same as in Figure 4 in the main text.
}\label{fig:A1B1_modifier_invasions}
\end{figure}


\end{document}  